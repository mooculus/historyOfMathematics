\documentclass[nooutcomes]{ximera}
\graphicspath{{./}{thePythagoreanTheorem/}{deMoivreSavesTheDay/}{complexNumbersFromDifferentAngles/}{trianglesOnACone/}{cityGeometry/}{EuclidAndGeometry/}}

\usepackage{gensymb}
\usepackage[margin=1in]{geometry}

%\usepackage{hyperref}


\usepackage{tikz}
\usepackage{tkz-euclide}
\usetkzobj{all}
\tikzstyle geometryDiagrams=[ultra thick,color=blue!50!black]
\newcommand{\tri}{\triangle}
\renewcommand{\l}{\ell}
\renewcommand{\P}{\mathcal{P}}
\newcommand{\R}{\mathbb{R}}
\newcommand{\Q}{\mathbb{Q}}

\newcommand{\Z}{\mathbb Z}
\newcommand{\N}{\mathbb N}
\newcommand{\ph}{\varphi}

\renewcommand{\vec}{\mathbf}
\renewcommand{\d}{\,d}


%\counterwithin*{question}{section} <- This didn't work



%% Egyptian symbols

\usepackage{multido}
\newcommand{\egmil}[1]{\multido{\i=1+1}{#1}{\includegraphics[scale=.1]{egyptian/egypt_person.pdf}\hspace{0.5mm}}}
\newcommand{\eghuntho}[1]{\multido{\i=1+1}{#1}{\includegraphics[scale=.1]{egyptian/egypt_fish.pdf}\hspace{0.5mm}}}
\newcommand{\egtentho}[1]{\multido{\i=1+1}{#1}{\includegraphics[scale=.1]{egyptian/egypt_finger.pdf}\hspace{0.5mm}}}
\newcommand{\egtho}[1]{\multido{\i=1+1}{#1}{\includegraphics[scale=.1]{egyptian/egypt_lotus.pdf}\hspace{0.5mm}}}
\newcommand{\eghun}[1]{\multido{\i=1+1}{#1}{\includegraphics[scale=.1]{egyptian/egypt_scroll.pdf}\hspace{0.5mm}}}
\newcommand{\egten}[1]{\multido{\i=1+1}{#1}{\includegraphics[scale=.1]{egyptian/egypt_heel.pdf}\hspace{0.5mm}}}
\newcommand{\egone}[1]{\multido{\i=1+1}{#1}{\includegraphics[scale=.1]{egyptian/egypt_stroke.pdf}\hspace{0.5mm}}}
\newcommand{\egyptify}[7]{
 \multido{\i=1+1}{#1}{\includegraphics[scale=.1]{egyptian/egypt_person.pdf}\hspace{0.5mm}}
 \multido{\i=1+1}{#2}{\includegraphics[scale=.1]{egyptian/egypt_fish.pdf}\hspace{0.5mm}}
 \multido{\i=1+1}{#3}{\includegraphics[scale=.1]{egyptian/egypt_finger.pdf}\hspace{0.5mm}}
 \multido{\i=1+1}{#4}{\includegraphics[scale=.1]{egyptian/egypt_lotus.pdf}\hspace{0.5mm}}
 \multido{\i=1+1}{#5}{\includegraphics[scale=.1]{egyptian/egypt_scroll.pdf}\hspace{0.5mm}}
 \multido{\i=1+1}{#6}{\includegraphics[scale=.1]{egyptian/egypt_heel.pdf}\hspace{0.5mm}}
 \multido{\i=1+1}{#7}{\includegraphics[scale=.1]{egyptian/egypt_stroke.pdf}\hspace{0.5mm}}
 \hspace{.5mm}
}




\title{Fermat's Last Theorem}

\begin{document}
\begin{abstract}
    
\end{abstract}
\maketitle

As we consider the Great Theorem of this chapter in our first reading,
we'll also talk a bit more about one of the mathematicians we have
mentioned a few times this semester: Pierre de Fermat.  Dunham discusses 
a little of Fermat's life story as well as Fermat's famous ``Last Theorem'' 
in the first part of Chapter 7. Dunham's text, however, is outdated on 
this subject: Fermat's Last Theorem has in fact been proven!  Our second
reading does a good job of explaining the basic ideas of the theorem
and the history of its proof.

The article in the second reading is still a little older than I'd like, 
however!  Here is a bit more information to bring our story into the present.
\begin{itemize}
	\item Since the article was published pretty close to
          the publication of Wiles' proof, it ends with some
          uncertainty about the proof.  That uncertainty has been
          removed - the proof has been reviewed and accepted.
	\item Wiles has since won a number of awards for his proof.  For example, in 2016 he received the \link[Abel Prize]{http://www.abelprize.no/} in Mathematics.  
	\item Wiles originally gave the proof in about three lectures; now 
	there are instances of entire courses being devoted to the matter.  
	If you're interested in seeing more details for the proof, you can find online anything from 
	\link[a relatively short summary]{https://math.bu.edu/people/ghs/papers/FermatOverview.pdf}, 
	to \link[an entire textbook]{https://library.ohio-state.edu/record=b4961885}!
	You will likely need to have had a significant abstract algebra course to understand much of the proof.
	\item The \link[Wikipedia page]{https://en.wikipedia.org/wiki/Fermat's_Last_Theorem} for Fermat's Last Theorem is heavily referenced and also not a bad place to get general information about the theorem.  (But as always with Wikipedia, it's good to confirm the information in a place that can't be publicly edited.)
\end{itemize}

Finally, the optional third reading (from ``Math with Bad Drawings'') offers us a more human perspective on this problem -- and perhaps other difficult problems as well.

\section{Readings}
First reading: Dunham, Chapter 7, pages 174 - 183

Second reading: \link[Fermat's Last Theorem]{https://mathshistory.st-andrews.ac.uk/HistTopics/Fermat's_last_theorem/}

Third reading: \link[The State of Being Stuck]{https://mathwithbaddrawings.com/2017/09/20/the-state-of-being-stuck/}



\section{Questions}

\begin{question}
Into how many cases did Sophie Germain divide the problem? $\answer[given]{2}$
\end{question}

\begin{question}
Where was Wiles working when he solved Fermat's Last Theorem?
\begin{multipleChoice}
\choice{Stanford.}
\choice [correct]{Princeton.}
\choice {Cambridge.}
\choice {Oxford.}
\end{multipleChoice}
\end{question}

%
%\begin{question}
%What are the most important points from this reading?
%\begin{freeResponse}
%\end{freeResponse}
%
%\end{question}


 %\link[a relatively short summary]{http://math.stanford.edu/~lekheng/flt/gouvea.pdf},

\end{document}
