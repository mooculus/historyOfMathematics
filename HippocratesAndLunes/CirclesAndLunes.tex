\documentclass{ximera}
\graphicspath{{./}{thePythagoreanTheorem/}{deMoivreSavesTheDay/}{complexNumbersFromDifferentAngles/}{trianglesOnACone/}{cityGeometry/}{EuclidAndGeometry/}}

\usepackage{gensymb}
\usepackage[margin=1in]{geometry}

%\usepackage{hyperref}


\usepackage{tikz}
\usepackage{tkz-euclide}
\usetkzobj{all}
\tikzstyle geometryDiagrams=[ultra thick,color=blue!50!black]
\newcommand{\tri}{\triangle}
\renewcommand{\l}{\ell}
\renewcommand{\P}{\mathcal{P}}
\newcommand{\R}{\mathbb{R}}
\newcommand{\Q}{\mathbb{Q}}

\newcommand{\Z}{\mathbb Z}
\newcommand{\N}{\mathbb N}
\newcommand{\ph}{\varphi}

\renewcommand{\vec}{\mathbf}
\renewcommand{\d}{\,d}


%\counterwithin*{question}{section} <- This didn't work



%% Egyptian symbols

\usepackage{multido}
\newcommand{\egmil}[1]{\multido{\i=1+1}{#1}{\includegraphics[scale=.1]{egyptian/egypt_person.pdf}\hspace{0.5mm}}}
\newcommand{\eghuntho}[1]{\multido{\i=1+1}{#1}{\includegraphics[scale=.1]{egyptian/egypt_fish.pdf}\hspace{0.5mm}}}
\newcommand{\egtentho}[1]{\multido{\i=1+1}{#1}{\includegraphics[scale=.1]{egyptian/egypt_finger.pdf}\hspace{0.5mm}}}
\newcommand{\egtho}[1]{\multido{\i=1+1}{#1}{\includegraphics[scale=.1]{egyptian/egypt_lotus.pdf}\hspace{0.5mm}}}
\newcommand{\eghun}[1]{\multido{\i=1+1}{#1}{\includegraphics[scale=.1]{egyptian/egypt_scroll.pdf}\hspace{0.5mm}}}
\newcommand{\egten}[1]{\multido{\i=1+1}{#1}{\includegraphics[scale=.1]{egyptian/egypt_heel.pdf}\hspace{0.5mm}}}
\newcommand{\egone}[1]{\multido{\i=1+1}{#1}{\includegraphics[scale=.1]{egyptian/egypt_stroke.pdf}\hspace{0.5mm}}}
\newcommand{\egyptify}[7]{
 \multido{\i=1+1}{#1}{\includegraphics[scale=.1]{egyptian/egypt_person.pdf}\hspace{0.5mm}}
 \multido{\i=1+1}{#2}{\includegraphics[scale=.1]{egyptian/egypt_fish.pdf}\hspace{0.5mm}}
 \multido{\i=1+1}{#3}{\includegraphics[scale=.1]{egyptian/egypt_finger.pdf}\hspace{0.5mm}}
 \multido{\i=1+1}{#4}{\includegraphics[scale=.1]{egyptian/egypt_lotus.pdf}\hspace{0.5mm}}
 \multido{\i=1+1}{#5}{\includegraphics[scale=.1]{egyptian/egypt_scroll.pdf}\hspace{0.5mm}}
 \multido{\i=1+1}{#6}{\includegraphics[scale=.1]{egyptian/egypt_heel.pdf}\hspace{0.5mm}}
 \multido{\i=1+1}{#7}{\includegraphics[scale=.1]{egyptian/egypt_stroke.pdf}\hspace{0.5mm}}
 \hspace{.5mm}
}




\title{Circles and Lunes}

\begin{document}
\begin{abstract}
\end{abstract}
\maketitle

In this section, we begin to read in our textbook for the course.  Dunham will begin from the beginning of mathematics as we know it, so much of the first reading will be a brief review of things we have studied so far.  Dunham's book is generally organized so that there is one ``Great Theorem'' in each chapter of the book.  The first of these is the earliest mathematical proof we have in its original form, by a Greek mathematician named Hippocrates.  

This result was not Hippocrates' only result, and he will play a role in the solutions to the other problems we will discuss in the next two sections.  The second reading looks at some of Hippocrates other results from a more modern perspective.  I encourage you to try some of the dynamic geometry exercises; as the article mentions, GeoGebra is free to use online.  You do not need to be able to recite the proofs of all of the lunes considered, but you should be able to give at least the proof of the Great Theorem in our textbook.




\section{Readings}

First Reading: Dunham, Chapter 1, pages 1 - 20

Second Reading: \link[Exploring the lunes of Hippocrates in a dynamic geometry environment]{http://dx.doi.org.proxy.lib.ohio-state.edu/10.1080/17498430.2015.1122301}




\section{Questions}

\begin{question}
How many types of lunes was Hippocrates able to square? $\answer[given]{3}$
\end{question}

\begin{question}
What does it mean to ``square'' a particular figure?
\begin{multipleChoice}
\choice{To cut up and rearrange the figure until it looks like a square.}
\choice{To use compass and straight edge to produce a triangle with the same area.}
\choice[correct]{To use compass and straight edge to produce any other figure with the same area which is known to be quadrable.}
\choice{To use compass and straight edge to construct a square whose corners intersect with the sides of the figure.}
\end{multipleChoice}
\end{question}


%
%\begin{question}
%What are the most important points of this reading?
%\begin{freeResponse}
%\end{freeResponse}
%\end{question}



\end{document}