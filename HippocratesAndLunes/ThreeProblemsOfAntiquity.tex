\documentclass{ximera}
\graphicspath{{./}{thePythagoreanTheorem/}{deMoivreSavesTheDay/}{complexNumbersFromDifferentAngles/}{trianglesOnACone/}{cityGeometry/}{EuclidAndGeometry/}}

\usepackage{gensymb}
\usepackage[margin=1in]{geometry}

%\usepackage{hyperref}


\usepackage{tikz}
\usepackage{tkz-euclide}
\usetkzobj{all}
\tikzstyle geometryDiagrams=[ultra thick,color=blue!50!black]
\newcommand{\tri}{\triangle}
\renewcommand{\l}{\ell}
\renewcommand{\P}{\mathcal{P}}
\newcommand{\R}{\mathbb{R}}
\newcommand{\Q}{\mathbb{Q}}

\newcommand{\Z}{\mathbb Z}
\newcommand{\N}{\mathbb N}
\newcommand{\ph}{\varphi}

\renewcommand{\vec}{\mathbf}
\renewcommand{\d}{\,d}



%\makeatletter
%\let\c@problem\relax
%\makeatother
%
%\let\problem\relax
%\let\endproblem\relax
%
%\newtheoremstyle{SlantTheorem}{\topsep}{\fill}%%% space between body and thm
% {\slshape}                      %%% Thm body font
% {}                              %%% Indent amount (empty = no indent)
% {\bfseries\sffamily}            %%% Thm head font
% {}                              %%% Punctuation after thm head
% {3ex}                           %%% Space after thm head
% {\thmname{#1}\thmnumber{ #2}\thmnote{ \bfseries(#3)}} %%% Thm head spec
%\theoremstyle{SlantTheorem}
%\newtheorem{problem}{Problem}[]
%
%
%\makeatletter
%\let\c@question\relax
%\makeatother
%
%\let\question\relax
%\let\endquestion\relax
%
%\newtheoremstyle{SlantTheorem}{\topsep}{\fill}%%% space between body and thm
% {\slshape}                      %%% Thm body font
% {}                              %%% Indent amount (empty = no indent)
% {\bfseries\sffamily}            %%% Thm head font
% {}                              %%% Punctuation after thm head
% {3ex}                           %%% Space after thm head
% {\thmname{#1}\thmnumber{ #2}\thmnote{ \bfseries(#3)}} %%% Thm head spec
%\theoremstyle{SlantTheorem}
%\newtheorem{question}{Question}[]






%\counterwithin*{question}{section} <- This didn't work



%% Egyptian symbols

\usepackage{multido}
\newcommand{\egmil}[1]{\multido{\i=1+1}{#1}{\includegraphics[scale=.1]{egyptian/egypt_person.pdf}\hspace{0.5mm}}}
\newcommand{\eghuntho}[1]{\multido{\i=1+1}{#1}{\includegraphics[scale=.1]{egyptian/egypt_fish.pdf}\hspace{0.5mm}}}
\newcommand{\egtentho}[1]{\multido{\i=1+1}{#1}{\includegraphics[scale=.1]{egyptian/egypt_finger.pdf}\hspace{0.5mm}}}
\newcommand{\egtho}[1]{\multido{\i=1+1}{#1}{\includegraphics[scale=.1]{egyptian/egypt_lotus.pdf}\hspace{0.5mm}}}
\newcommand{\eghun}[1]{\multido{\i=1+1}{#1}{\includegraphics[scale=.1]{egyptian/egypt_scroll.pdf}\hspace{0.5mm}}}
\newcommand{\egten}[1]{\multido{\i=1+1}{#1}{\includegraphics[scale=.1]{egyptian/egypt_heel.pdf}\hspace{0.5mm}}}
\newcommand{\egone}[1]{\multido{\i=1+1}{#1}{\includegraphics[scale=.1]{egyptian/egypt_stroke.pdf}\hspace{0.5mm}}}
\newcommand{\egyptify}[7]{
 \multido{\i=1+1}{#1}{\includegraphics[scale=.1]{egyptian/egypt_person.pdf}\hspace{0.5mm}}
 \multido{\i=1+1}{#2}{\includegraphics[scale=.1]{egyptian/egypt_fish.pdf}\hspace{0.5mm}}
 \multido{\i=1+1}{#3}{\includegraphics[scale=.1]{egyptian/egypt_finger.pdf}\hspace{0.5mm}}
 \multido{\i=1+1}{#4}{\includegraphics[scale=.1]{egyptian/egypt_lotus.pdf}\hspace{0.5mm}}
 \multido{\i=1+1}{#5}{\includegraphics[scale=.1]{egyptian/egypt_scroll.pdf}\hspace{0.5mm}}
 \multido{\i=1+1}{#6}{\includegraphics[scale=.1]{egyptian/egypt_heel.pdf}\hspace{0.5mm}}
 \multido{\i=1+1}{#7}{\includegraphics[scale=.1]{egyptian/egypt_stroke.pdf}\hspace{0.5mm}}
 \hspace{.5mm}
}




\title{The Three Problems of Antiquity}

\begin{document}
\begin{abstract}
\end{abstract}
\maketitle

Hippocrates' squaring of a particular lune in the Great Theorem of our first chapter of Dunham's book is just the beginning of a discussion about three famous problems on which ancient Greek mathematicians worked.  These three problems are:
\begin{enumerate}
	\item Can every circle be squared using only compass and straightedge?
	\item Can every cube be doubled using only compass and straightedge?
	\item Can every angle be trisected using only compass and straightedge?
\end{enumerate}
We will find that the answer to each of these questions is actually ``no'', and that the negative answer is potentially the more interesting answer for this question.  As the Greek mathematicians explored these problems, they also began to change the questions slightly as they asked them.  Changing the question slightly in order to get a related answer is a very common mathematical method of today, and one of the best ways to begin to chip away at a difficult mathematical problem.  By the time we are finished studying these problems, you should be able to talk about why each question cannot be solved with compass and straightedge alone, and you should be able to discuss at least one solution to each problem not using compass and straightedge.

In this section, we will explore briefly the solution to the circle squaring problem, and we will also look at the cube doubling problem.  In the next section, we will tackle the angle trisection problem.



\section{Readings}

First Reading: Dunham, Chapter 1, pages 20 - 26

Second Reading: \link[Doubling the Cube]{http://www-history.mcs.st-and.ac.uk/HistTopics/Doubling_the_cube.html}

Third Reading: \link[A Method of Duplicating the Cube]{https://www.maa.org/sites/default/files/Graef32917.pdf}



\section{Questions}

\begin{question}
How many stories are told in the readings about how the cube doubling problem originated?  $\answer[given]{2}$
\end{question}



\begin{question}
What geometric subject did Menaechmus discover while trying to duplicate the cube?
\begin{multipleChoice}
\choice[correct]{Conic sections.}
\choice{Calculus.}
\choice{Formulas for volumes of pyramids.}
\choice{Formulas for slopes of lines.}
\end{multipleChoice}
\end{question}


%\begin{question}
%What are the most important points of this reading?
%\begin{freeResponse}
%\end{freeResponse}
%\end{question}



\end{document}
