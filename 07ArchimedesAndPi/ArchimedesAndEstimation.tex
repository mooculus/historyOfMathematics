\documentclass[nooutcomes]{ximera}
\graphicspath{{./}{thePythagoreanTheorem/}{deMoivreSavesTheDay/}{complexNumbersFromDifferentAngles/}{trianglesOnACone/}{cityGeometry/}{EuclidAndGeometry/}}

\usepackage{gensymb}
\usepackage[margin=1in]{geometry}

%\usepackage{hyperref}


\usepackage{tikz}
\usepackage{tkz-euclide}
\usetkzobj{all}
\tikzstyle geometryDiagrams=[ultra thick,color=blue!50!black]
\newcommand{\tri}{\triangle}
\renewcommand{\l}{\ell}
\renewcommand{\P}{\mathcal{P}}
\newcommand{\R}{\mathbb{R}}
\newcommand{\Q}{\mathbb{Q}}

\newcommand{\Z}{\mathbb Z}
\newcommand{\N}{\mathbb N}
\newcommand{\ph}{\varphi}

\renewcommand{\vec}{\mathbf}
\renewcommand{\d}{\,d}



%\makeatletter
%\let\c@problem\relax
%\makeatother
%
%\let\problem\relax
%\let\endproblem\relax
%
%\newtheoremstyle{SlantTheorem}{\topsep}{\fill}%%% space between body and thm
% {\slshape}                      %%% Thm body font
% {}                              %%% Indent amount (empty = no indent)
% {\bfseries\sffamily}            %%% Thm head font
% {}                              %%% Punctuation after thm head
% {3ex}                           %%% Space after thm head
% {\thmname{#1}\thmnumber{ #2}\thmnote{ \bfseries(#3)}} %%% Thm head spec
%\theoremstyle{SlantTheorem}
%\newtheorem{problem}{Problem}[]
%
%
%\makeatletter
%\let\c@question\relax
%\makeatother
%
%\let\question\relax
%\let\endquestion\relax
%
%\newtheoremstyle{SlantTheorem}{\topsep}{\fill}%%% space between body and thm
% {\slshape}                      %%% Thm body font
% {}                              %%% Indent amount (empty = no indent)
% {\bfseries\sffamily}            %%% Thm head font
% {}                              %%% Punctuation after thm head
% {3ex}                           %%% Space after thm head
% {\thmname{#1}\thmnumber{ #2}\thmnote{ \bfseries(#3)}} %%% Thm head spec
%\theoremstyle{SlantTheorem}
%\newtheorem{question}{Question}[]






%\counterwithin*{question}{section} <- This didn't work



%% Egyptian symbols

\usepackage{multido}
\newcommand{\egmil}[1]{\multido{\i=1+1}{#1}{\includegraphics[scale=.1]{egyptian/egypt_person.pdf}\hspace{0.5mm}}}
\newcommand{\eghuntho}[1]{\multido{\i=1+1}{#1}{\includegraphics[scale=.1]{egyptian/egypt_fish.pdf}\hspace{0.5mm}}}
\newcommand{\egtentho}[1]{\multido{\i=1+1}{#1}{\includegraphics[scale=.1]{egyptian/egypt_finger.pdf}\hspace{0.5mm}}}
\newcommand{\egtho}[1]{\multido{\i=1+1}{#1}{\includegraphics[scale=.1]{egyptian/egypt_lotus.pdf}\hspace{0.5mm}}}
\newcommand{\eghun}[1]{\multido{\i=1+1}{#1}{\includegraphics[scale=.1]{egyptian/egypt_scroll.pdf}\hspace{0.5mm}}}
\newcommand{\egten}[1]{\multido{\i=1+1}{#1}{\includegraphics[scale=.1]{egyptian/egypt_heel.pdf}\hspace{0.5mm}}}
\newcommand{\egone}[1]{\multido{\i=1+1}{#1}{\includegraphics[scale=.1]{egyptian/egypt_stroke.pdf}\hspace{0.5mm}}}
\newcommand{\egyptify}[7]{
 \multido{\i=1+1}{#1}{\includegraphics[scale=.1]{egyptian/egypt_person.pdf}\hspace{0.5mm}}
 \multido{\i=1+1}{#2}{\includegraphics[scale=.1]{egyptian/egypt_fish.pdf}\hspace{0.5mm}}
 \multido{\i=1+1}{#3}{\includegraphics[scale=.1]{egyptian/egypt_finger.pdf}\hspace{0.5mm}}
 \multido{\i=1+1}{#4}{\includegraphics[scale=.1]{egyptian/egypt_lotus.pdf}\hspace{0.5mm}}
 \multido{\i=1+1}{#5}{\includegraphics[scale=.1]{egyptian/egypt_scroll.pdf}\hspace{0.5mm}}
 \multido{\i=1+1}{#6}{\includegraphics[scale=.1]{egyptian/egypt_heel.pdf}\hspace{0.5mm}}
 \multido{\i=1+1}{#7}{\includegraphics[scale=.1]{egyptian/egypt_stroke.pdf}\hspace{0.5mm}}
 \hspace{.5mm}
}




\title{Archimedes and Estimation}

\begin{document}
\begin{abstract}
    
\end{abstract}
\maketitle

We are now introduced to perhaps the greatest mathematician of antiquity, Archimedes.  There are many things for which he is famous, but for our Great Theorem, Dunham has chosen his proof that the area formula for circles is $A = \pi r^2$.  Of course, Archimedes won't write the formula quite this way, and we will see what he does to get around the notation.  As we look at the Great Theorem of this chapter, one question many people have when they look at Archimedes' work is how he came to his conclusions.  In particular, Archimedes spent a lot of time estimating values that today we would usually find using a calculator.  There are a number of different ways to estimate things like $\pi$ and square roots.  In the second reading, we consider some ways that Archimedes might have arrived at his conclusions.  For the second reading, you should read at least the introduction and Section 4.


\section{Readings}
First reading: Dunham, Chapter 4, pages 84 - 99

Second reading: \link[Archimedes' calculations of square roots]{https://arxiv.org/abs/1101.0492}

\section{Questions}

\begin{question}
The fourth step in the interpolation method places $\sqrt{3}$ between $\answer[given]{5/3}$ and $\answer[given]{7/4}$
\end{question}

\begin{question}
The author believes that the interpolation method shows that Archimedes understood what topic?
\begin{multipleChoice}
\choice{Astronomy}
\choice {Derivatives}
\choice {Geometry}
\choice [correct]{Limits}
\end{multipleChoice}
\end{question}


%\begin{question}
%What are the most important points from this reading?
%\begin{freeResponse}
%\end{freeResponse}

%\end{question}




\end{document}
