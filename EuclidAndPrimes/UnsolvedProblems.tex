\documentclass[nooutcomes]{ximera}

\title{Unsolved Problems}

\begin{document}
\begin{abstract}
    
\end{abstract}
\maketitle



Our next chapter is the first where we'll discuss number theory, a branch of mathematics wherein we study properties of whole numbers and relationships between them.  Number theory is sometimes used as an introduction to ``higher mathematics'', because the definitions are usually easy to grasp.  Examples are often easy to come by, and conjectures seem to follow naturally.  

Once we begin to ask questions about the relationships between numbers, we quickly realize that many of these questions are incredibly easy to state, but incredibly difficult to prove!  In fact, some of the most well-known unsolved problems in mathematics come from the branch of number theory.

In the second reading, we'll be introduced to some famous unsolved problems in mathematics.  Some of these problems are mentioned in the first reading, and some of them are not.  As you look at this article, make sure to click on some of the links to get more information about topics that look interesting to you.  The Collatz problem, for instance, is very easy to try out for yourself.  The notion that 10 is a solitary number is related to some things we read about the Pythagoreans, and of course we've already discussed the case of odd perfect numbers.


\section{Readings}
First reading: Dunham, Chapter 3, pages 61-73

Second reading: \link[Difficult Problems]{http://mathworld.wolfram.com/UnsolvedProblems.html}.

In the second reading, please click on the links on the page to get more information about these problems.  You should read at least the following.
\begin{enumerate}
    \item \link[The Goldbach Conjecture]{http://mathworld.wolfram.com/GoldbachConjecture.html}
    \item \link[Twin Primes]{http://mathworld.wolfram.com/TwinPrimes.html}
    \item \link[The Twin Prime Conjecture]{http://mathworld.wolfram.com/TwinPrimeConjecture.html}
\end{enumerate}



\section{Questions}

\begin{question}
How many versions or types of Goldbach's conjecture are listed in the article on that topic?
$\answer[given]{7}$
\end{question}

\begin{question}
Which of the following are twin primes?
\begin{multipleChoice}
\choice{$1$ and $3$}
\choice {$11$ and $23$}
\choice {$23$ and $46$}
\choice [correct]{$29$ and $31$}
\end{multipleChoice}
\end{question}


\begin{question}
What are the most important points from this reading?
\begin{freeResponse}
\end{freeResponse}

\end{question}




\end{document}
