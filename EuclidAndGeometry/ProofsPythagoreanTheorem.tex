\documentclass[nooutcomes]{ximera}

\title{Proofs of the Pythagorean Theorem}

\begin{document}
\begin{abstract}
    
\end{abstract}
\maketitle


We will study Euclid for two chapters - the first focused on geometry and the second focused on number theory.  Euclid's name is worth knowing because of his work called the ``Elements'', where he attempts to construct all of the mathematics known at the time from basic assumptions he calls ``common notions'' and ``postulates''.  By the time we are finished with this chapter, you should be able to state Euclid's Fifth Postulate and say something about why it was controversial.  

Two important people who influenced Euclid's thinking are \link[Eudoxus]{http://www-groups.dcs.st-and.ac.uk/history/Biographies/Eudoxus.html}, most famous for his ``method of exhaustion'', and \link[Aristotle]{http://www-groups.dcs.st-and.ac.uk/history/Biographies/Aristotle.html} who wrote on what proof in mathematics should be, and may have been the first to use the phrase ``common notion''.


\section{Readings}
First reading: Dunham Chapter 2

Second Reading: \link[Proofs of the Pythagorean Theorem]{http://www.cut-the-knot.org/pythagoras/}
  

In the second reading, you should read the introduction, and then pick a few of these proofs to study.  You do not need to know all of the proofs on this site!  You should be able to give, in full detail, the proof from our textbook (which is also Proof \#1 on the site) as well as two other proofs of your choice.


\section{Questions}

\begin{question}
How many proofs are listed on this site?
$\answer[given]{118}$
\end{question}

\begin{question}
Which of the following is NOT a category of proofs of the theorem mentioned in the remarks?
\begin{multipleChoice}
\choice[correct] {Proofs by contradiction.}
\choice {Algebraic proofs.}
\choice {Geometric proofs.}
\choice {Trigonometric proofs.}
\end{multipleChoice}
\end{question}


\begin{question}
What are the most important points from this reading?
\begin{freeResponse}
\end{freeResponse}

\end{question}

\end{document}