\documentclass{ximera}

\graphicspath{{./}{thePythagoreanTheorem/}{deMoivreSavesTheDay/}{complexNumbersFromDifferentAngles/}{trianglesOnACone/}{cityGeometry/}{EuclidAndGeometry/}}

\usepackage{gensymb}
\usepackage[margin=1in]{geometry}

%\usepackage{hyperref}


\usepackage{tikz}
\usepackage{tkz-euclide}
\usetkzobj{all}
\tikzstyle geometryDiagrams=[ultra thick,color=blue!50!black]
\newcommand{\tri}{\triangle}
\renewcommand{\l}{\ell}
\renewcommand{\P}{\mathcal{P}}
\newcommand{\R}{\mathbb{R}}
\newcommand{\Q}{\mathbb{Q}}

\newcommand{\Z}{\mathbb Z}
\newcommand{\N}{\mathbb N}
\newcommand{\ph}{\varphi}

\renewcommand{\vec}{\mathbf}
\renewcommand{\d}{\,d}


%\counterwithin*{question}{section} <- This didn't work



%% Egyptian symbols

\usepackage{multido}
\newcommand{\egmil}[1]{\multido{\i=1+1}{#1}{\includegraphics[scale=.1]{egyptian/egypt_person.pdf}\hspace{0.5mm}}}
\newcommand{\eghuntho}[1]{\multido{\i=1+1}{#1}{\includegraphics[scale=.1]{egyptian/egypt_fish.pdf}\hspace{0.5mm}}}
\newcommand{\egtentho}[1]{\multido{\i=1+1}{#1}{\includegraphics[scale=.1]{egyptian/egypt_finger.pdf}\hspace{0.5mm}}}
\newcommand{\egtho}[1]{\multido{\i=1+1}{#1}{\includegraphics[scale=.1]{egyptian/egypt_lotus.pdf}\hspace{0.5mm}}}
\newcommand{\eghun}[1]{\multido{\i=1+1}{#1}{\includegraphics[scale=.1]{egyptian/egypt_scroll.pdf}\hspace{0.5mm}}}
\newcommand{\egten}[1]{\multido{\i=1+1}{#1}{\includegraphics[scale=.1]{egyptian/egypt_heel.pdf}\hspace{0.5mm}}}
\newcommand{\egone}[1]{\multido{\i=1+1}{#1}{\includegraphics[scale=.1]{egyptian/egypt_stroke.pdf}\hspace{0.5mm}}}
\newcommand{\egyptify}[7]{
 \multido{\i=1+1}{#1}{\includegraphics[scale=.1]{egyptian/egypt_person.pdf}\hspace{0.5mm}}
 \multido{\i=1+1}{#2}{\includegraphics[scale=.1]{egyptian/egypt_fish.pdf}\hspace{0.5mm}}
 \multido{\i=1+1}{#3}{\includegraphics[scale=.1]{egyptian/egypt_finger.pdf}\hspace{0.5mm}}
 \multido{\i=1+1}{#4}{\includegraphics[scale=.1]{egyptian/egypt_lotus.pdf}\hspace{0.5mm}}
 \multido{\i=1+1}{#5}{\includegraphics[scale=.1]{egyptian/egypt_scroll.pdf}\hspace{0.5mm}}
 \multido{\i=1+1}{#6}{\includegraphics[scale=.1]{egyptian/egypt_heel.pdf}\hspace{0.5mm}}
 \multido{\i=1+1}{#7}{\includegraphics[scale=.1]{egyptian/egypt_stroke.pdf}\hspace{0.5mm}}
 \hspace{.5mm}
}




\title{Euclid's Elements}

\begin{document}
\begin{abstract}
We prove some propositions from Euclid's \textit{Elements}. \end{abstract}

\maketitle



A full list of the definitions, common notions, postulates, and propositions can be found at \url{http://aleph0.clarku.edu/~djoyce/java/elements/bookI/bookI.html}


\begin{proposition}[I.5]
In isosceles triangles the angles at the base equal one another, and, if the equal straight lines are produced further, then the angles under the base equal one another.
\end{proposition}

\begin{question}
Prove Proposition I.5.
\end{question}

\begin{proposition}[I.6]
If in a triangle two angles equal one another, then the sides opposite the equal angles also equal one another.
\end{proposition}

\begin{question}
Prove Proposition I.6.
\end{question}


\begin{proposition}[I.15]
If two straight-lines cut one another then they make the vertically
opposite angles equal to one another.
\end{proposition}

\begin{question}
Prove Proposition I.15.
\end{question}

\begin{proposition}[I.16]
For any triangle, when one of the sides is produced, the external
angle is greater than each of the internal and opposite angles.
\begin{image}
\begin{tikzpicture}[geometryDiagrams]
\coordinate (A) at (0,2);
\coordinate (B) at (2,5);
\coordinate (C) at (6.5,2);
\coordinate (F) at (9,2);
\draw (A)--(B)--(C)--cycle;
\draw (C)--(F);
\tkzLabelPoints[above](B)
\tkzLabelPoints[below](A,C)
\tkzMarkAngle[size=0.5cm,thin](F,C,B)
\tkzLabelAngle[pos = 0.25](F,C,B){$\epsilon$}

\tkzMarkAngle[size=0.6cm,thin](A,B,C)
\tkzLabelAngle[pos = 0.35](A,B,C){$\beta$}

\tkzMarkAngle[size=0.6cm,thin](C,A,B)
\tkzLabelAngle[pos = 0.35](C,A,B){$\alpha$}

%\draw[step=.5cm] (0,0) grid (10,5);
\end{tikzpicture}
\end{image}
\end{proposition}

\begin{question}
Prove Proposition I.16.
\end{question}

\begin{proposition}[I.27]
If a straight-line falling across two straight-lines makes the
alternate angles equal to one another then the (two) straight-lines
will be parallel to one another.
\end{proposition}


\begin{question}
Prove Proposition I.27.
\end{question}


\begin{proposition}[I.29]
A straight-line falling across parallel straight-lines makes the
alternate angles equal to one another, the external (angle) equal to
the internal and opposite (angle), and the (sum of the) internal
(angles) on the same side equal to two right-angles.
\end{proposition}


\begin{question}
Prove Proposition I.29.
\end{question}


\begin{proposition}[I.32]
In any triangle, (if) one of the sides (is) produced (then) the
external angle is equal to the (sum of the) two internal and opposite
(angles), and the (sum of the) three internal angles of the triangle
is equal to two right-angles.
\end{proposition}


\begin{question}
Prove Proposition I.32.
\end{question}






\end{document}