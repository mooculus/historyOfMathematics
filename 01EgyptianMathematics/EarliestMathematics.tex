\documentclass{ximera}
\graphicspath{{./}{thePythagoreanTheorem/}{deMoivreSavesTheDay/}{complexNumbersFromDifferentAngles/}{trianglesOnACone/}{cityGeometry/}{EuclidAndGeometry/}}

\usepackage{gensymb}
\usepackage[margin=1in]{geometry}

%\usepackage{hyperref}


\usepackage{tikz}
\usepackage{tkz-euclide}
\usetkzobj{all}
\tikzstyle geometryDiagrams=[ultra thick,color=blue!50!black]
\newcommand{\tri}{\triangle}
\renewcommand{\l}{\ell}
\renewcommand{\P}{\mathcal{P}}
\newcommand{\R}{\mathbb{R}}
\newcommand{\Q}{\mathbb{Q}}

\newcommand{\Z}{\mathbb Z}
\newcommand{\N}{\mathbb N}
\newcommand{\ph}{\varphi}

\renewcommand{\vec}{\mathbf}
\renewcommand{\d}{\,d}


%\counterwithin*{question}{section} <- This didn't work



%% Egyptian symbols

\usepackage{multido}
\newcommand{\egmil}[1]{\multido{\i=1+1}{#1}{\includegraphics[scale=.1]{egyptian/egypt_person.pdf}\hspace{0.5mm}}}
\newcommand{\eghuntho}[1]{\multido{\i=1+1}{#1}{\includegraphics[scale=.1]{egyptian/egypt_fish.pdf}\hspace{0.5mm}}}
\newcommand{\egtentho}[1]{\multido{\i=1+1}{#1}{\includegraphics[scale=.1]{egyptian/egypt_finger.pdf}\hspace{0.5mm}}}
\newcommand{\egtho}[1]{\multido{\i=1+1}{#1}{\includegraphics[scale=.1]{egyptian/egypt_lotus.pdf}\hspace{0.5mm}}}
\newcommand{\eghun}[1]{\multido{\i=1+1}{#1}{\includegraphics[scale=.1]{egyptian/egypt_scroll.pdf}\hspace{0.5mm}}}
\newcommand{\egten}[1]{\multido{\i=1+1}{#1}{\includegraphics[scale=.1]{egyptian/egypt_heel.pdf}\hspace{0.5mm}}}
\newcommand{\egone}[1]{\multido{\i=1+1}{#1}{\includegraphics[scale=.1]{egyptian/egypt_stroke.pdf}\hspace{0.5mm}}}
\newcommand{\egyptify}[7]{
 \multido{\i=1+1}{#1}{\includegraphics[scale=.1]{egyptian/egypt_person.pdf}\hspace{0.5mm}}
 \multido{\i=1+1}{#2}{\includegraphics[scale=.1]{egyptian/egypt_fish.pdf}\hspace{0.5mm}}
 \multido{\i=1+1}{#3}{\includegraphics[scale=.1]{egyptian/egypt_finger.pdf}\hspace{0.5mm}}
 \multido{\i=1+1}{#4}{\includegraphics[scale=.1]{egyptian/egypt_lotus.pdf}\hspace{0.5mm}}
 \multido{\i=1+1}{#5}{\includegraphics[scale=.1]{egyptian/egypt_scroll.pdf}\hspace{0.5mm}}
 \multido{\i=1+1}{#6}{\includegraphics[scale=.1]{egyptian/egypt_heel.pdf}\hspace{0.5mm}}
 \multido{\i=1+1}{#7}{\includegraphics[scale=.1]{egyptian/egypt_stroke.pdf}\hspace{0.5mm}}
 \hspace{.5mm}
}




\title{Earliest Mathematics}

\begin{document}
\begin{abstract}
\end{abstract}
\maketitle

Welcome to Math 4504! Before we get started, here are a few notes.
\begin{itemize}
	\item The readings we will have generally come from two places: the book ``Journey Through Genius'' by William Dunham, and online.
	\item The online readings should all be free.  For many of them, hosted through the library, you will need to use your OSU login to access the resource. (If you are already logged in to another resource like Carmen, you may automatically bypass the login step!)  If you run into trouble with any of the resources, please check Carmen or contact me.
	\item Remember to access this site through Carmen so that your score translates back to the grade book appropriately.
\end{itemize}

Now, on to the good stuff!


We begin our study of the history of mathematics as far back in history as we can.  The earliest form of mathematics that we know is counting, as our 
ancestors worked to keep track of how many of various things they had. The earliest evidence of counting we have is a prehistoric bone on which have been 
marked some tallies, which sometimes appear to be in groups of five.  You can see a picture of these marks on what is now called the ``Ishango bone'' 
at \link[Prehistoric Mathematics]{http://www.storyofmathematics.com/prehistoric.html}.  The earliest civilization we know to then develop methods of 
adding, subtracting, multiplying, and dividing are the ancient Egyptians.  In the readings below, we will see some history of the time period, as well as 
the methods the Egyptians used for counting and basic mathematical operations.  The third reading is a timeline, which you might find helpful during 
this first part of our course.




\section{Readings}

First Reading: (Video) \link[The Language of the Universe: Mathematics in Ancient Times]{http://library.ohio-state.edu/record=b7179127~S7}
\begin{itemize}
\item Watch Section 1: Emergence of a New Universe (Approx 3min)
\item Watch Section 2: Egyptian Numbers (Approx 5 min)
\item Watch Section 3: Mathematics in Everyday Egyptian Life (Approx 5 min)
\end{itemize}

%Second Reading: \link[Egyptian Mathematics]{http://www.math.tamu.edu/~don.allen/history/egypt/egypt.html}
%\begin{itemize}
%\item Section 1: Basic Facts About Ancient Egypt
%\item Section 2: Counting and Arithmetic: Basics
%\end{itemize}

Second Reading (optional): Some examples of Egyptian calculations. \link[Egyptian Arithmetic]{http://www.math.buffalo.edu/mad/Ancient-Africa/mad_ancient_egypt_arith.html}

Third Reading: \link[Egyptian Fractions: Ahmes to Fibonacci to Today]{https://www.jstor.org/stable/27967280}

Fourth Reading: \link[Chronology for 30000BC to 500BC]{https://mathshistory.st-andrews.ac.uk/Chronology/1/}


\section{Questions}



\begin{question}
When are the first symbols for numbers used?  $\answer[given]{3400}$BC
\end{question}



\begin{question}
What kind of fractions did the Ancient Egyptians use?
\begin{multipleChoice}
\choice{They did not use fractions.}
\choice[correct]{Unit fractions $\frac{1}{n}$}
\choice{Only the fractions $\frac12$, $\frac23$, and $\frac34$}
\choice{All fractions that we use today.}
\end{multipleChoice}
\end{question}



%\begin{question}
%What are the most important points of this reading?
%\begin{freeResponse}
%\end{freeResponse}
%\end{question}



\end{document}
