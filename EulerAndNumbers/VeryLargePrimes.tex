\documentclass[nooutcomes]{ximera}

\title{Very Large Primes}

\begin{document}
\begin{abstract}
    
\end{abstract}
\maketitle

Being one of the most prolific mathematicians in history, Euler considered many subjects.  Here, we look at Euler's work with some of Fermat's conjectures in number theory.  In an unusual turn of events, we consider the Great Theorem as part of the first reading for this chapter.  One of the consequences of this Great Theorem is that people began using Euler's proof method to look for larger and larger primes.  This was one of the first ways to look for primes other than simply considering all of their factors.  Different methods have since been invented, so Euler's method isn't necessarily used anymore, but our second reading explores ``records'' for the largest known primes.  The article also does a nice job detailing how things changed after the modern computer was invented!  You can also click the links in the article if you're interested in learning more about how people search for primes.


\section{Readings}
First reading: Dunham, Chapter 10, pages 223 - 235

Second reading: \link[The Largest Known Primes by Year]{http://primes.utm.edu/notes/by_year.html}

\begin{itemize}
	\item The current largest known prime was found on December 26, 2017.  See this \link[press release]{https://www.mersenne.org/primes/press/M77232917.html} for more details!
\end{itemize}



\section{Questions}

\begin{question}
How many digits are in the largest prime mentioned in the article? $\answer[given]{23249425}$
\end{question}

\begin{question}
Who has the record for the largest prime found without using a machine?
\begin{multipleChoice}
\choice[correct]{Lucas}
\choice {Euler}
\choice {Ferrier}
\choice {Miller \& Wheeler}
\end{multipleChoice}
\end{question}


\begin{question}
What are the most important points from this reading?
\begin{freeResponse}
\end{freeResponse}

\end{question}




\end{document}
