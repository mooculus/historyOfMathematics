\documentclass[nooutcomes]{ximera}
\graphicspath{{./}{thePythagoreanTheorem/}{deMoivreSavesTheDay/}{complexNumbersFromDifferentAngles/}{trianglesOnACone/}{cityGeometry/}{EuclidAndGeometry/}}

\usepackage{gensymb}
\usepackage[margin=1in]{geometry}

%\usepackage{hyperref}


\usepackage{tikz}
\usepackage{tkz-euclide}
\usetkzobj{all}
\tikzstyle geometryDiagrams=[ultra thick,color=blue!50!black]
\newcommand{\tri}{\triangle}
\renewcommand{\l}{\ell}
\renewcommand{\P}{\mathcal{P}}
\newcommand{\R}{\mathbb{R}}
\newcommand{\Q}{\mathbb{Q}}

\newcommand{\Z}{\mathbb Z}
\newcommand{\N}{\mathbb N}
\newcommand{\ph}{\varphi}

\renewcommand{\vec}{\mathbf}
\renewcommand{\d}{\,d}


%\counterwithin*{question}{section} <- This didn't work



%% Egyptian symbols

\usepackage{multido}
\newcommand{\egmil}[1]{\multido{\i=1+1}{#1}{\includegraphics[scale=.1]{egyptian/egypt_person.pdf}\hspace{0.5mm}}}
\newcommand{\eghuntho}[1]{\multido{\i=1+1}{#1}{\includegraphics[scale=.1]{egyptian/egypt_fish.pdf}\hspace{0.5mm}}}
\newcommand{\egtentho}[1]{\multido{\i=1+1}{#1}{\includegraphics[scale=.1]{egyptian/egypt_finger.pdf}\hspace{0.5mm}}}
\newcommand{\egtho}[1]{\multido{\i=1+1}{#1}{\includegraphics[scale=.1]{egyptian/egypt_lotus.pdf}\hspace{0.5mm}}}
\newcommand{\eghun}[1]{\multido{\i=1+1}{#1}{\includegraphics[scale=.1]{egyptian/egypt_scroll.pdf}\hspace{0.5mm}}}
\newcommand{\egten}[1]{\multido{\i=1+1}{#1}{\includegraphics[scale=.1]{egyptian/egypt_heel.pdf}\hspace{0.5mm}}}
\newcommand{\egone}[1]{\multido{\i=1+1}{#1}{\includegraphics[scale=.1]{egyptian/egypt_stroke.pdf}\hspace{0.5mm}}}
\newcommand{\egyptify}[7]{
 \multido{\i=1+1}{#1}{\includegraphics[scale=.1]{egyptian/egypt_person.pdf}\hspace{0.5mm}}
 \multido{\i=1+1}{#2}{\includegraphics[scale=.1]{egyptian/egypt_fish.pdf}\hspace{0.5mm}}
 \multido{\i=1+1}{#3}{\includegraphics[scale=.1]{egyptian/egypt_finger.pdf}\hspace{0.5mm}}
 \multido{\i=1+1}{#4}{\includegraphics[scale=.1]{egyptian/egypt_lotus.pdf}\hspace{0.5mm}}
 \multido{\i=1+1}{#5}{\includegraphics[scale=.1]{egyptian/egypt_scroll.pdf}\hspace{0.5mm}}
 \multido{\i=1+1}{#6}{\includegraphics[scale=.1]{egyptian/egypt_heel.pdf}\hspace{0.5mm}}
 \multido{\i=1+1}{#7}{\includegraphics[scale=.1]{egyptian/egypt_stroke.pdf}\hspace{0.5mm}}
 \hspace{.5mm}
}




\title{Proofs that the Harmonic Series Diverges}

\begin{document}
\begin{abstract}
    
\end{abstract}
\maketitle

Our Great Theorem of Chapter 8 is Johann Bernoulli's proof that the Harmonic Series diverges.  We'll talk about why this is a surprising result, as well as some other attempts that were made at the proof, particularly by Leibniz.  

As we saw with the Pythagorean Theorem, mathematicians often like to find several different ways of proving theorems - particularly when the result of the theorem is surprising or important.  Sometimes this seems silly to non-mathematicians, since the result of the theorem is already known.  But using different methods of proof can help us understand connections we've never before seen between seemingly different areas of mathematics, or shed light on related problems that are not yet solved.

Again, as with the Pythagorean Theorem, you are not expected to know all of the proofs given in the articles below.  Choose one or two other than the proof given as the Great Theorem.



\section{Readings}
First reading: Dunham, Chapter 8, pages 196-206

Second reading: \link[The Harmonic Series Diverges Again and Again]{http://scipp.ucsc.edu/~haber/archives/physics116A10/harmapa.pdf}.

Third reading: \link[An Exceedingly Short Proof that the Harmonic Series Diverges]{https://projecteuclid.org/journals/missouri-journal-of-mathematical-sciences/volume-27/issue-1/An-Exceedingly-Short-Proof-that-the-Harmonic-Series-Diverges/10.35834/mjms/1449161372.full}



\section{Questions}
In the list of proofs that the Harmonic Series diverges, in what year was the earliest one given?
\begin{question}

$\answer[given]{1350}$
\end{question}

\begin{question}
The ``exceedingly short proof'' is done by what method?
\begin{multipleChoice}
\choice{Guessing.}
\choice {Computing an integral.}
\choice {Evaluating the series directly.}
\choice [correct]{Comparison with another series.}
\end{multipleChoice}
\end{question}


%\begin{question}
%What are the most important points from this reading?
%\begin{freeResponse}
%\end{freeResponse}
%
%\end{question}




\end{document}
