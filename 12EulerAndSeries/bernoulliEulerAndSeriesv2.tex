\documentclass[nooutcomes]{ximera}

\graphicspath{{./}{thePythagoreanTheorem/}{deMoivreSavesTheDay/}{complexNumbersFromDifferentAngles/}{trianglesOnACone/}{cityGeometry/}{EuclidAndGeometry/}}

\usepackage{gensymb}
\usepackage[margin=1in]{geometry}

%\usepackage{hyperref}


\usepackage{tikz}
\usepackage{tkz-euclide}
\usetkzobj{all}
\tikzstyle geometryDiagrams=[ultra thick,color=blue!50!black]
\newcommand{\tri}{\triangle}
\renewcommand{\l}{\ell}
\renewcommand{\P}{\mathcal{P}}
\newcommand{\R}{\mathbb{R}}
\newcommand{\Q}{\mathbb{Q}}

\newcommand{\Z}{\mathbb Z}
\newcommand{\N}{\mathbb N}
\newcommand{\ph}{\varphi}

\renewcommand{\vec}{\mathbf}
\renewcommand{\d}{\,d}


%\counterwithin*{question}{section} <- This didn't work



%% Egyptian symbols

\usepackage{multido}
\newcommand{\egmil}[1]{\multido{\i=1+1}{#1}{\includegraphics[scale=.1]{egyptian/egypt_person.pdf}\hspace{0.5mm}}}
\newcommand{\eghuntho}[1]{\multido{\i=1+1}{#1}{\includegraphics[scale=.1]{egyptian/egypt_fish.pdf}\hspace{0.5mm}}}
\newcommand{\egtentho}[1]{\multido{\i=1+1}{#1}{\includegraphics[scale=.1]{egyptian/egypt_finger.pdf}\hspace{0.5mm}}}
\newcommand{\egtho}[1]{\multido{\i=1+1}{#1}{\includegraphics[scale=.1]{egyptian/egypt_lotus.pdf}\hspace{0.5mm}}}
\newcommand{\eghun}[1]{\multido{\i=1+1}{#1}{\includegraphics[scale=.1]{egyptian/egypt_scroll.pdf}\hspace{0.5mm}}}
\newcommand{\egten}[1]{\multido{\i=1+1}{#1}{\includegraphics[scale=.1]{egyptian/egypt_heel.pdf}\hspace{0.5mm}}}
\newcommand{\egone}[1]{\multido{\i=1+1}{#1}{\includegraphics[scale=.1]{egyptian/egypt_stroke.pdf}\hspace{0.5mm}}}
\newcommand{\egyptify}[7]{
 \multido{\i=1+1}{#1}{\includegraphics[scale=.1]{egyptian/egypt_person.pdf}\hspace{0.5mm}}
 \multido{\i=1+1}{#2}{\includegraphics[scale=.1]{egyptian/egypt_fish.pdf}\hspace{0.5mm}}
 \multido{\i=1+1}{#3}{\includegraphics[scale=.1]{egyptian/egypt_finger.pdf}\hspace{0.5mm}}
 \multido{\i=1+1}{#4}{\includegraphics[scale=.1]{egyptian/egypt_lotus.pdf}\hspace{0.5mm}}
 \multido{\i=1+1}{#5}{\includegraphics[scale=.1]{egyptian/egypt_scroll.pdf}\hspace{0.5mm}}
 \multido{\i=1+1}{#6}{\includegraphics[scale=.1]{egyptian/egypt_heel.pdf}\hspace{0.5mm}}
 \multido{\i=1+1}{#7}{\includegraphics[scale=.1]{egyptian/egypt_stroke.pdf}\hspace{0.5mm}}
 \hspace{.5mm}
}




\title{Bernoulli, Euler, and series}

\begin{document}
\begin{abstract}
Here we see some topics that both Bernoulli and Euler were interested in.
\end{abstract}
\maketitle



Finding the sum of the following series is called ``The Basel Problem'' as it interested several mathematicians with connections to the city of Basel, Switzerland.  (Who were they?)
\[
1 + \frac{1}{4} + \frac{1}{9} + \frac{1}{16} + \frac{1}{25} + \frac{1}{36} + \frac{1}{49}\cdots
\]
Notice: we are asking for the sum of the reciprocals of the square numbers. 


\begin{question}
Consider:
\[
f(x) = 1 - \frac{x^2}{3!} + \frac{x^4}{5!}-\frac{x^6}{7!} + \frac{x^8}{9!} - \frac{x^{10}}{11!} + \dots
\]
Can you explain why 
\[
f(x) = \frac{\sin(x)}{x}\qquad x \ne 0?
\]
\end{question}


\begin{question}
Let $g(x)$ be a polynomial with roots $a_1,\dots, a_n$. Suppose also
that $g(0)=0$. What are the factors of $g(x)$?
\end{question}

\begin{question}
Let $g(x)$ be a polynomial with roots $a_1,\dots, a_n$. Suppose also
that $g(0)=1$. What are the factors of $g(x)$?
\end{question}

\begin{question}
What exactly are the roots of $f(x)$? What is $f(0)$?  Explain why:
\[
f(x) = \bigg(1-\frac{x}{\pi} \bigg)\bigg(1-\frac{x}{-\pi} \bigg)\bigg(1-\frac{x}{2\pi} \bigg)\bigg(1-\frac{x}{-2\pi} \bigg)\bigg(1-\frac{x}{3\pi} \bigg)\bigg(1-\frac{x}{-3\pi} \bigg) \cdots
\]
\end{question}

\begin{question}
Explain why:
\[
f(x) = \prod_{n=1}^\infty \left(1 - \frac{x^2}{n^2\pi^2} \right)
\]
\end{question}

\begin{question}
Explain why:
\[
f(x) = 1 - x^2 \sum_{n=1}^\infty \frac{1}{n^2\pi^2} 
+ x^4 \bigg(\cdots\bigg)  - x^6 \bigg(\cdots\bigg)  + \cdots
\]
\end{question}


\begin{question}
Explain why:
\[
\sum_{n=1}^\infty \frac{1}{n^2}= \frac{\pi^2}{6}.
\]
\end{question}

\begin{exploration}[Bonus!]
Compute
\[
\sum_{n=1}^\infty \frac{1}{n^3}.
\]
\end{exploration}

\end{document}
