\documentclass[handout,nooutcomes]{ximera}
\graphicspath{{./}{thePythagoreanTheorem/}{deMoivreSavesTheDay/}{complexNumbersFromDifferentAngles/}{trianglesOnACone/}{cityGeometry/}{EuclidAndGeometry/}}

\usepackage{gensymb}
\usepackage[margin=1in]{geometry}

%\usepackage{hyperref}


\usepackage{tikz}
\usepackage{tkz-euclide}
\usetkzobj{all}
\tikzstyle geometryDiagrams=[ultra thick,color=blue!50!black]
\newcommand{\tri}{\triangle}
\renewcommand{\l}{\ell}
\renewcommand{\P}{\mathcal{P}}
\newcommand{\R}{\mathbb{R}}
\newcommand{\Q}{\mathbb{Q}}

\newcommand{\Z}{\mathbb Z}
\newcommand{\N}{\mathbb N}
\newcommand{\ph}{\varphi}

\renewcommand{\vec}{\mathbf}
\renewcommand{\d}{\,d}



%\makeatletter
%\let\c@problem\relax
%\makeatother
%
%\let\problem\relax
%\let\endproblem\relax
%
%\newtheoremstyle{SlantTheorem}{\topsep}{\fill}%%% space between body and thm
% {\slshape}                      %%% Thm body font
% {}                              %%% Indent amount (empty = no indent)
% {\bfseries\sffamily}            %%% Thm head font
% {}                              %%% Punctuation after thm head
% {3ex}                           %%% Space after thm head
% {\thmname{#1}\thmnumber{ #2}\thmnote{ \bfseries(#3)}} %%% Thm head spec
%\theoremstyle{SlantTheorem}
%\newtheorem{problem}{Problem}[]
%
%
%\makeatletter
%\let\c@question\relax
%\makeatother
%
%\let\question\relax
%\let\endquestion\relax
%
%\newtheoremstyle{SlantTheorem}{\topsep}{\fill}%%% space between body and thm
% {\slshape}                      %%% Thm body font
% {}                              %%% Indent amount (empty = no indent)
% {\bfseries\sffamily}            %%% Thm head font
% {}                              %%% Punctuation after thm head
% {3ex}                           %%% Space after thm head
% {\thmname{#1}\thmnumber{ #2}\thmnote{ \bfseries(#3)}} %%% Thm head spec
%\theoremstyle{SlantTheorem}
%\newtheorem{question}{Question}[]






%\counterwithin*{question}{section} <- This didn't work



%% Egyptian symbols

\usepackage{multido}
\newcommand{\egmil}[1]{\multido{\i=1+1}{#1}{\includegraphics[scale=.1]{egyptian/egypt_person.pdf}\hspace{0.5mm}}}
\newcommand{\eghuntho}[1]{\multido{\i=1+1}{#1}{\includegraphics[scale=.1]{egyptian/egypt_fish.pdf}\hspace{0.5mm}}}
\newcommand{\egtentho}[1]{\multido{\i=1+1}{#1}{\includegraphics[scale=.1]{egyptian/egypt_finger.pdf}\hspace{0.5mm}}}
\newcommand{\egtho}[1]{\multido{\i=1+1}{#1}{\includegraphics[scale=.1]{egyptian/egypt_lotus.pdf}\hspace{0.5mm}}}
\newcommand{\eghun}[1]{\multido{\i=1+1}{#1}{\includegraphics[scale=.1]{egyptian/egypt_scroll.pdf}\hspace{0.5mm}}}
\newcommand{\egten}[1]{\multido{\i=1+1}{#1}{\includegraphics[scale=.1]{egyptian/egypt_heel.pdf}\hspace{0.5mm}}}
\newcommand{\egone}[1]{\multido{\i=1+1}{#1}{\includegraphics[scale=.1]{egyptian/egypt_stroke.pdf}\hspace{0.5mm}}}
\newcommand{\egyptify}[7]{
 \multido{\i=1+1}{#1}{\includegraphics[scale=.1]{egyptian/egypt_person.pdf}\hspace{0.5mm}}
 \multido{\i=1+1}{#2}{\includegraphics[scale=.1]{egyptian/egypt_fish.pdf}\hspace{0.5mm}}
 \multido{\i=1+1}{#3}{\includegraphics[scale=.1]{egyptian/egypt_finger.pdf}\hspace{0.5mm}}
 \multido{\i=1+1}{#4}{\includegraphics[scale=.1]{egyptian/egypt_lotus.pdf}\hspace{0.5mm}}
 \multido{\i=1+1}{#5}{\includegraphics[scale=.1]{egyptian/egypt_scroll.pdf}\hspace{0.5mm}}
 \multido{\i=1+1}{#6}{\includegraphics[scale=.1]{egyptian/egypt_heel.pdf}\hspace{0.5mm}}
 \multido{\i=1+1}{#7}{\includegraphics[scale=.1]{egyptian/egypt_stroke.pdf}\hspace{0.5mm}}
 \hspace{.5mm}
}




\title{Twentieth Century Mathematicians}



\begin{document}
\begin{abstract}
    
\end{abstract}
\maketitle

Your goal for today is to learn about some more recent mathematicians. For each 
of your mathematicians, write down the following information.
\begin{itemize}
\item When they live(d)
\item Where they live(d)
\item What they study or studied (you don't have to understand their topic!)
\item An interesting fact about them
\end{itemize}

After some work time, each group will present. If there are $N$ people in your group, 
I'd like to see you talk about at least $N$ mathematicians.

At the end of class, we'll fill out a participation assignment where you'll write the 
names of two mathematicians: one who has something similar to you, and one who has 
something different from you. The similarities and differences are up to you!

Any mathematician born after the year 1900 is valid for this exercise. I'll list some 
names below if you need some ideas to get started, but I encourage you to also find people 
who are not on this list!
	
  \begin{itemize}
\item Artur Avila
\item David Blackwell
\item Ron Buckmeyer
\item James Donaldson
\item Joan Clarke
\item Paul Erd\H{o}s
\item Etta Zuber Falconer
\item Kurt G\"{o}del
\item Jes\'{u}s De Loera Herrera
\item June Huh
\item Katherine Johnson
\item Beno\^{i}t Mandelbrot
\item Maryam Mirzakhani
\item Katherine Okikiolu
\item Grigori Perelman
\item Emily Riehl
\item Julia Robinson
\item Jean Springer
\item Thomas Storer (b. 1938)
\item Hortensia Soto
\item Terrence Tao
\item Alan Turing
\item John Urschel
\end{itemize}

\end{document}
