\documentclass{ximera}

\title{Babylonian Basics}

\begin{document}
\begin{abstract}
\end{abstract}
\maketitle

The next ancient civilization we will study are the ancient Babylonians.  As with the Egyptians, we will begin with their number system and ways of doing arithmetic, then move on in the next section to problem-solving methods.  The two readings below have some overlap, which will hopefully help to clarify the material.  Especially when it comes to these ancient civilizations, it's also good to have several perspectives, as historians' opinions can differ quite wildly.  For instance, historians have long thought that the tablet called ``Plimpton 322'' was a list of Pythagorean Triples, and used this evidence to cite that the ancient Babylonians knew this theorem long before Pythagoras.  However, historians are still looking at this tablet to determine whether this conclusion is accurate.  An article whose author takes a different interpretation is \link[Words and Pictures: New Light on Plimpton 322]{http://www.jstor.org/stable/2695324}.  An even more recent article (August 2017) takes a different approach still: \link[Mathematical secrets of ancient tablet unlocked after nearly a century of study]{https://www.theguardian.com/science/2017/aug/24/mathematical-secrets-of-ancient-tablet-unlocked-after-nearly-a-century-of-study?CMP=fb_gu}.

If after you read these selections and still have some questions, the MacTutor articles on this topic may also be helpful.  In particular, the Babylonian method of division is often very confusing!  The optional third reading about reciprocals of numbers addresses this topic.

MacTutor articles (optional):
\begin{itemize}
\item \link[An overview of Babylonian mathematics]{http://www-history.mcs.st-and.ac.uk/HistTopics/Babylonian_mathematics.html}
\item \link[Babylonian numerals]{http://www-history.mcs.st-and.ac.uk/HistTopics/Babylonian_numerals.html}
\end{itemize}




\section{Readings}

First Reading: \link[Counting in Cuneiform]{http://www.jstor.org.proxy.lib.ohio-state.edu/stable/30211866} 

Second Reading: \link[Babylonian Mathematics]{http://www.math.tamu.edu/~dallen/masters/egypt_babylon/babylon.pdf}

Optional Third Reading: \link[Babylonian Mathematical Texts I: Reciprocals of Regular Sexagesimal Numbers]{http://www.jstor.org.proxy.lib.ohio-state.edu/stable/1359434}


\section{Questions}

\begin{question}
What base did the Babylonians use for their number system? $\answer[given]{60}$
\end{question}

\begin{question}
Why did the Babylonians make tables?  Choose the best answer.
\begin{multipleChoice}
\choice{For fun.}
\choice{For use by non-mathematicians.}
\choice{To demonstrate solutions to problems.}
\choice[correct]{To simplify calculations.}
\end{multipleChoice}
\end{question}



\begin{question}
What are the most important points of this reading?
\begin{freeResponse}
\end{freeResponse}
\end{question}



\end{document}
