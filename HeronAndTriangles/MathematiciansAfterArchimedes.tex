\documentclass[nooutcomes]{ximera}

\title{Mathematicians after Archimedes}

\begin{document}
\begin{abstract}
    
\end{abstract}
\maketitle

Archimedes was certainly a mathematical genius, and following in his footsteps would certainly have been challenging.  Our subject now is these mathematicians, living in changing times.  It is relatively fitting to the story of mathematics that Archimedes was killed by the Romans, as the Roman takeover of much of the Greek world changed the way that people thought about and did mathematics for many years.

Here are some names of mathematicians whose contributions we will discuss at least briefly in class and in our readings.  By the end of our time on this chapter, you should be able to say a few words about what each person did.
\begin{itemize}
\item Eratosthenes (the subject of our second reading)
\item Apollonius
\item Heron (the subject of our first reading)
\item Ptolemy
\item Hypatia
\item Diophantus
\end{itemize}




\section{Readings}
First reading: Dunham, Chapter 5, pages 113 - 121 

Second reading: \link[Roman Mathematics]{http://www.storyofmathematics.com/roman.html}

Third reading (short video): \link[The Language of the Universe: Mathematics in Ancient Times]{http://library.ohio-state.edu/record=b7179127~S7} \\ Watch Section 15: End Times of Greek Mathematics

Fourth reading: \link[Diophantus]{http://www.storyofmathematics.com/hellenistic_diophantus.html}

\section{Questions}

\begin{question}
According to the epitaph riddle, how old was Diophantus when he died? $\answer[given]{84}$
\end{question}

\begin{question}
How did the Romans generally do their calculations?
\begin{multipleChoice}
\choice{After praying to a pagan god.}
\choice {Converting to base 60 and then solve.}
\choice {Writing an abstract equation and solving.}
\choice [correct]{Using an abacus.}
\end{multipleChoice}
\end{question}


\begin{question}
What are the most important points from this reading?
\begin{freeResponse}
\end{freeResponse}

\end{question}




\end{document}
