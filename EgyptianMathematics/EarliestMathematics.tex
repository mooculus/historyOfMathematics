\documentclass{ximera}

\title{Earliest Mathematics}

\begin{document}
\begin{abstract}
\end{abstract}
\maketitle

We begin our study of the history of mathematics as far back in history as we can.  The earliest form of mathematics that we know is counting, as our 
ancestors worked to keep track of how many of various things they had. and the earliest evidence we have is a prehistoric bone on which have been 
marked some tallies, which sometimes appear to be in groups of five.  You can see a picture of these marks on what is now called the ``Ishago bone'' 
at \link[Prehistoric Mathematics]{http://www.storyofmathematics.com/prehistoric.html}.  The earliest civilization we know to then develop methods of 
adding, subtracting, multiplying, and dividing are the ancient Egyptians.  In the readings below, we will see some history of the time period, as well as 
the methods the Egyptians used for counting and basic mathematical operations.  The third reading is a timeline, which you might find helpful during 
this first part of our course.




\section{Readings}

First Reading: (Video) \link[The Language of the Universe: Mathematics in Ancient Times]{http://fod.infobase.com.proxy.lib.ohio-state.edu/p_ViewVideo.aspx?xtid=40029}
\begin{itemize}
\item Watch at least Section 1: Emergence of a New Universe
\end{itemize}

Second Reading: \link[Egyptian Mathematics]{http://www.math.tamu.edu/~don.allen/history/egypt/egypt.html}
\begin{itemize}
\item Section 1: Basic Facts About Ancient Egypt
\item Section 2: Counting and Arithmetic: Basics
\end{itemize}

Third Reading: \link[Egyptian Fractions: Ahmes to Fibonacci to Today]{http://www.jstor.org.proxy.lib.ohio-state.edu/stable/27967280}

Fourth Reading: \link[Chronology for 30000BC to 500BC]{http://www-history.mcs.st-and.ac.uk/Chronology/30000BC_500BC.html}


\section{Questions}



\begin{question}
When are the first symbols for numbers used?  $\answer[given]{3400}$BC
\end{question}



\begin{question}
What kind of fractions did the Ancient Egyptians use?
\begin{multipleChoice}
\choice{They did not use fractions.}
\choice[correct]{Unit fractions $\frac{1}{n}$}
\choice{Only the fractions $\frac12$, $\frac23$, and $\frac34$}
\choice{All fractions that we use today.}
\end{multipleChoice}
\end{question}



\begin{question}
What are the most important points of this reading?
\begin{freeResponse}
\end{freeResponse}
\end{question}



\end{document}
