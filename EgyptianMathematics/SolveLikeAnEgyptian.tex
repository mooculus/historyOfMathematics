\documentclass{ximera}
\graphicspath{{./}{thePythagoreanTheorem/}{deMoivreSavesTheDay/}{complexNumbersFromDifferentAngles/}{trianglesOnACone/}{cityGeometry/}{EuclidAndGeometry/}}

\usepackage{gensymb}
\usepackage[margin=1in]{geometry}

%\usepackage{hyperref}


\usepackage{tikz}
\usepackage{tkz-euclide}
\usetkzobj{all}
\tikzstyle geometryDiagrams=[ultra thick,color=blue!50!black]
\newcommand{\tri}{\triangle}
\renewcommand{\l}{\ell}
\renewcommand{\P}{\mathcal{P}}
\newcommand{\R}{\mathbb{R}}
\newcommand{\Q}{\mathbb{Q}}

\newcommand{\Z}{\mathbb Z}
\newcommand{\N}{\mathbb N}
\newcommand{\ph}{\varphi}

\renewcommand{\vec}{\mathbf}
\renewcommand{\d}{\,d}


%\counterwithin*{question}{section} <- This didn't work



%% Egyptian symbols

\usepackage{multido}
\newcommand{\egmil}[1]{\multido{\i=1+1}{#1}{\includegraphics[scale=.1]{egyptian/egypt_person.pdf}\hspace{0.5mm}}}
\newcommand{\eghuntho}[1]{\multido{\i=1+1}{#1}{\includegraphics[scale=.1]{egyptian/egypt_fish.pdf}\hspace{0.5mm}}}
\newcommand{\egtentho}[1]{\multido{\i=1+1}{#1}{\includegraphics[scale=.1]{egyptian/egypt_finger.pdf}\hspace{0.5mm}}}
\newcommand{\egtho}[1]{\multido{\i=1+1}{#1}{\includegraphics[scale=.1]{egyptian/egypt_lotus.pdf}\hspace{0.5mm}}}
\newcommand{\eghun}[1]{\multido{\i=1+1}{#1}{\includegraphics[scale=.1]{egyptian/egypt_scroll.pdf}\hspace{0.5mm}}}
\newcommand{\egten}[1]{\multido{\i=1+1}{#1}{\includegraphics[scale=.1]{egyptian/egypt_heel.pdf}\hspace{0.5mm}}}
\newcommand{\egone}[1]{\multido{\i=1+1}{#1}{\includegraphics[scale=.1]{egyptian/egypt_stroke.pdf}\hspace{0.5mm}}}
\newcommand{\egyptify}[7]{
 \multido{\i=1+1}{#1}{\includegraphics[scale=.1]{egyptian/egypt_person.pdf}\hspace{0.5mm}}
 \multido{\i=1+1}{#2}{\includegraphics[scale=.1]{egyptian/egypt_fish.pdf}\hspace{0.5mm}}
 \multido{\i=1+1}{#3}{\includegraphics[scale=.1]{egyptian/egypt_finger.pdf}\hspace{0.5mm}}
 \multido{\i=1+1}{#4}{\includegraphics[scale=.1]{egyptian/egypt_lotus.pdf}\hspace{0.5mm}}
 \multido{\i=1+1}{#5}{\includegraphics[scale=.1]{egyptian/egypt_scroll.pdf}\hspace{0.5mm}}
 \multido{\i=1+1}{#6}{\includegraphics[scale=.1]{egyptian/egypt_heel.pdf}\hspace{0.5mm}}
 \multido{\i=1+1}{#7}{\includegraphics[scale=.1]{egyptian/egypt_stroke.pdf}\hspace{0.5mm}}
 \hspace{.5mm}
}




\title{Solve Like an Egyptian}

\begin{document}
\begin{abstract}
\end{abstract}
\maketitle



We have seen how the ancient Egyptians worked through basic arithmetic problems.  Our next question should then be: what kinds of problems did they solve?  Generally, the ancient Egyptians are known for solving practical, every-day problems that had to do with administering their large empire.  Scribes would solve geometry and arithmetic problems as part of their jobs, and other people in the empire would generally not know or use such mathematics.  Scribes were trained in scribal schools, where they learned mathematics.  The best examples we have come from papyrus rolls which have been preserved and translated.  Our goal for the readings below is to understand some examples of how the ancient Egyptians solved arithmetic and geometric problems.



\section{Readings}

First Reading: \link[Mathematics Problems from Ancient Egyptian Papyri]{http://www.jstor.org.proxy.lib.ohio-state.edu/stable/20876630}

Second Reading: \link[Mathematics in Egyptian Papyri]{http://www-history.mcs.st-and.ac.uk/HistTopics/Egyptian_papyri.html}

%Third Reading: \link[The Moscow Papyrus]{http://www.math.tamu.edu/~don.allen/history/egypt/node4.html}

Third Reading: \link[Summary of Egyptian Mathematics]{http://www.math.tamu.edu/~don.allen/history/egypt/node5.html}


\section{Questions}

\begin{question}
What was the Egyptian value for $\pi$? $\answer[given]{3.1605}$
\end{question}

\begin{question}
What is the first step in the Method of False Position?
\begin{multipleChoice}
\choice{Moving everything to one side of the equals sign.}
\choice{Labeling a variable.}
\choice[correct]{Guessing.}
\choice{Doubling numbers using a chart.}
\end{multipleChoice}
\end{question}


%\begin{question}
%What are the most important points of this reading?
%\begin{freeResponse}
%\end{freeResponse}
%\end{question}



\end{document}
