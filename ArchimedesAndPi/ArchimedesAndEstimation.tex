\documentclass[nooutcomes]{ximera}

\title{Archimedes and Estimation}

\begin{document}
\begin{abstract}
    
\end{abstract}
\maketitle

As we look at the Great Theorem of this chapter, one question many people have when they look at Archimedes' work is how he came to his conclusions.  In particular, Archimedes spent a lot of time estimating values that today we would usually find using a calculator.  There are a number of different ways to estimate things like $\pi$, which we talked about in the previous readings, and square roots.  In the second reading, we consider some ways that Archimedes might have arrived at his conclusions.  For the second reading, you should read at least the introduction and Section 4.


\section{Readings}
First reading: Dunham, Chapter 4, pages 99 - 112

Second reading: \link[Archimedes' calculations of square roots]{https://arxiv.org/abs/1101.0492}

\section{Questions}

\begin{question}
The fourth step in the interpolation method places $\sqrt{3}$ between $\answer[given]{5/3}$ and $\answer[given]{7/4}$
\end{question}

\begin{question}
The author believes that the interpolation method shows that Archimedes understood what topic?
\begin{multipleChoice}
\choice{Astronomy}
\choice {Derivatives}
\choice {Geometry}
\choice [correct]{Limits}
\end{multipleChoice}
\end{question}


\begin{question}
What are the most important points from this reading?
\begin{freeResponse}
\end{freeResponse}

\end{question}




\end{document}
