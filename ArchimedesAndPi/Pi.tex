\documentclass[nooutcomes]{ximera}
\graphicspath{{./}{thePythagoreanTheorem/}{deMoivreSavesTheDay/}{complexNumbersFromDifferentAngles/}{trianglesOnACone/}{cityGeometry/}{EuclidAndGeometry/}}

\usepackage{gensymb}
\usepackage[margin=1in]{geometry}

%\usepackage{hyperref}


\usepackage{tikz}
\usepackage{tkz-euclide}
\usetkzobj{all}
\tikzstyle geometryDiagrams=[ultra thick,color=blue!50!black]
\newcommand{\tri}{\triangle}
\renewcommand{\l}{\ell}
\renewcommand{\P}{\mathcal{P}}
\newcommand{\R}{\mathbb{R}}
\newcommand{\Q}{\mathbb{Q}}

\newcommand{\Z}{\mathbb Z}
\newcommand{\N}{\mathbb N}
\newcommand{\ph}{\varphi}

\renewcommand{\vec}{\mathbf}
\renewcommand{\d}{\,d}


%\counterwithin*{question}{section} <- This didn't work



%% Egyptian symbols

\usepackage{multido}
\newcommand{\egmil}[1]{\multido{\i=1+1}{#1}{\includegraphics[scale=.1]{egyptian/egypt_person.pdf}\hspace{0.5mm}}}
\newcommand{\eghuntho}[1]{\multido{\i=1+1}{#1}{\includegraphics[scale=.1]{egyptian/egypt_fish.pdf}\hspace{0.5mm}}}
\newcommand{\egtentho}[1]{\multido{\i=1+1}{#1}{\includegraphics[scale=.1]{egyptian/egypt_finger.pdf}\hspace{0.5mm}}}
\newcommand{\egtho}[1]{\multido{\i=1+1}{#1}{\includegraphics[scale=.1]{egyptian/egypt_lotus.pdf}\hspace{0.5mm}}}
\newcommand{\eghun}[1]{\multido{\i=1+1}{#1}{\includegraphics[scale=.1]{egyptian/egypt_scroll.pdf}\hspace{0.5mm}}}
\newcommand{\egten}[1]{\multido{\i=1+1}{#1}{\includegraphics[scale=.1]{egyptian/egypt_heel.pdf}\hspace{0.5mm}}}
\newcommand{\egone}[1]{\multido{\i=1+1}{#1}{\includegraphics[scale=.1]{egyptian/egypt_stroke.pdf}\hspace{0.5mm}}}
\newcommand{\egyptify}[7]{
 \multido{\i=1+1}{#1}{\includegraphics[scale=.1]{egyptian/egypt_person.pdf}\hspace{0.5mm}}
 \multido{\i=1+1}{#2}{\includegraphics[scale=.1]{egyptian/egypt_fish.pdf}\hspace{0.5mm}}
 \multido{\i=1+1}{#3}{\includegraphics[scale=.1]{egyptian/egypt_finger.pdf}\hspace{0.5mm}}
 \multido{\i=1+1}{#4}{\includegraphics[scale=.1]{egyptian/egypt_lotus.pdf}\hspace{0.5mm}}
 \multido{\i=1+1}{#5}{\includegraphics[scale=.1]{egyptian/egypt_scroll.pdf}\hspace{0.5mm}}
 \multido{\i=1+1}{#6}{\includegraphics[scale=.1]{egyptian/egypt_heel.pdf}\hspace{0.5mm}}
 \multido{\i=1+1}{#7}{\includegraphics[scale=.1]{egyptian/egypt_stroke.pdf}\hspace{0.5mm}}
 \hspace{.5mm}
}




\title{Pi}

\begin{document}
\begin{abstract}
    
\end{abstract}
\maketitle

We are now introduced to perhaps the greatest mathematician of antiquity, Archimedes.  There are many things for which he is famous, but for our Great Theorem, Dunham has chosen his proof that the area formula for circles is $A = \pi r^2$.  Of course, Archimedes won't write the formula quite this way, and we will see what he does to get around the notation.  Another circle-related result for which Archimedes is quite famous is his approximation for $\pi$, which we still sometimes use today.  Our second reading will be a history of this constant throughout history.  People are still studying $\pi$ today!  For more information about modern estimations of $\pi$, see the third reading.  The optional fourth reading has more detail than the second one.

In case you are curious: the current (as of July 2017) record for digits of $\pi$ computed is 22.4 trillion.  See ``Chronology of computation of $\pi$'' on Wikipedia (and its linked sources) for an overview.


\section{Readings}
First reading: Dunham, Chapter 4, pages 84 - 99

Second reading: \link[A History of Pi]{http://www-groups.dcs.st-and.ac.uk/~history/HistTopics/Pi_through_the_ages.html}

Third reading: \link[12.1 Trillion Digits of Pi]{http://www.numberworld.org/misc_runs/pi-12t/}

Fourth reading: \link[$\pi$: A Brief History]{http://www.math.tamu.edu/~dallen/masters/alg_numtheory/pi.pdf}

\section{Questions}

\begin{question}
How long did the computation of 12.1 trillion digits of $\pi$ take? $\answer[given]{94}$ days
\end{question}

\begin{question}
Which of the following had the most accurate estimation for $\pi$?
\begin{multipleChoice}
\choice{Archimedes}
\choice {Ptolemy}
\choice [correct]{Zu Chongzhi}
\choice {al-Khwarizmi}
\end{multipleChoice}
\end{question}


%\begin{question}
%What are the most important points from this reading?
%\begin{freeResponse}
%\end{freeResponse}
%
%\end{question}




\end{document}
