\documentclass[nooutcomes]{ximera}
\graphicspath{{./}{thePythagoreanTheorem/}{deMoivreSavesTheDay/}{complexNumbersFromDifferentAngles/}{trianglesOnACone/}{cityGeometry/}{EuclidAndGeometry/}}

\usepackage{gensymb}
\usepackage[margin=1in]{geometry}

%\usepackage{hyperref}


\usepackage{tikz}
\usepackage{tkz-euclide}
\usetkzobj{all}
\tikzstyle geometryDiagrams=[ultra thick,color=blue!50!black]
\newcommand{\tri}{\triangle}
\renewcommand{\l}{\ell}
\renewcommand{\P}{\mathcal{P}}
\newcommand{\R}{\mathbb{R}}
\newcommand{\Q}{\mathbb{Q}}

\newcommand{\Z}{\mathbb Z}
\newcommand{\N}{\mathbb N}
\newcommand{\ph}{\varphi}

\renewcommand{\vec}{\mathbf}
\renewcommand{\d}{\,d}


%\counterwithin*{question}{section} <- This didn't work



%% Egyptian symbols

\usepackage{multido}
\newcommand{\egmil}[1]{\multido{\i=1+1}{#1}{\includegraphics[scale=.1]{egyptian/egypt_person.pdf}\hspace{0.5mm}}}
\newcommand{\eghuntho}[1]{\multido{\i=1+1}{#1}{\includegraphics[scale=.1]{egyptian/egypt_fish.pdf}\hspace{0.5mm}}}
\newcommand{\egtentho}[1]{\multido{\i=1+1}{#1}{\includegraphics[scale=.1]{egyptian/egypt_finger.pdf}\hspace{0.5mm}}}
\newcommand{\egtho}[1]{\multido{\i=1+1}{#1}{\includegraphics[scale=.1]{egyptian/egypt_lotus.pdf}\hspace{0.5mm}}}
\newcommand{\eghun}[1]{\multido{\i=1+1}{#1}{\includegraphics[scale=.1]{egyptian/egypt_scroll.pdf}\hspace{0.5mm}}}
\newcommand{\egten}[1]{\multido{\i=1+1}{#1}{\includegraphics[scale=.1]{egyptian/egypt_heel.pdf}\hspace{0.5mm}}}
\newcommand{\egone}[1]{\multido{\i=1+1}{#1}{\includegraphics[scale=.1]{egyptian/egypt_stroke.pdf}\hspace{0.5mm}}}
\newcommand{\egyptify}[7]{
 \multido{\i=1+1}{#1}{\includegraphics[scale=.1]{egyptian/egypt_person.pdf}\hspace{0.5mm}}
 \multido{\i=1+1}{#2}{\includegraphics[scale=.1]{egyptian/egypt_fish.pdf}\hspace{0.5mm}}
 \multido{\i=1+1}{#3}{\includegraphics[scale=.1]{egyptian/egypt_finger.pdf}\hspace{0.5mm}}
 \multido{\i=1+1}{#4}{\includegraphics[scale=.1]{egyptian/egypt_lotus.pdf}\hspace{0.5mm}}
 \multido{\i=1+1}{#5}{\includegraphics[scale=.1]{egyptian/egypt_scroll.pdf}\hspace{0.5mm}}
 \multido{\i=1+1}{#6}{\includegraphics[scale=.1]{egyptian/egypt_heel.pdf}\hspace{0.5mm}}
 \multido{\i=1+1}{#7}{\includegraphics[scale=.1]{egyptian/egypt_stroke.pdf}\hspace{0.5mm}}
 \hspace{.5mm}
}




\title{Logarithms and Multiplication}

\begin{document}
\begin{abstract}
    
\end{abstract}
\maketitle

In the late 1500s and early 1600s, astronomy was really taking off.  (Kepler, for instance, lived from 1571 - 1630.)  The more accurate astronomy became, the more mathematicians needed to do computations with ever-larger numbers.  The invention of the logarithm (published in 1614 and 1619) by John Napier, with assistance from Henry Briggs, was designed to shorten these calculations.  In Napier's own words: 

{\em Seeing there is nothing (right well-beloved Students of the Mathematics) that is so troublesome to mathematical practice, nor that doth more molest and hinder calculators, than the multiplications, divisions, square and cubical extractions of great numbers, which besides the tedious expense of time are for the most part subject to many slippery errors, I began therefore to consider in my mind by what certain and ready art I might remove those hindrances. And having thought upon many things to this purpose, I found at length some excellent brief rules to be treated of (perhaps) hereafter. But amongst all, none more profitable than this which together with the hard and tedious multiplications, divisions, and extractions of roots, doth also cast away from the work itself even the very numbers themselves that are to be multiplied, divided and resolved into roots, and putteth other numbers in their place which perform as much as they can do, only by addition and subtraction, division by two or division by three.}

Our second reading is a video explaining some ways we know to speed up multiplication, including the use of a logarithm.  These methods are an important backdrop to our first reading, which is about the life and mathematics of Isaac Newton.

Sources (and optional further reading):
\begin{itemize}
\item \link[Napier Biography]{https://mathshistory.st-andrews.ac.uk/Biographies/Napier/}
\item \link[John Napier: His Life, His Logs, and His Bones]{https://www.maa.org/press/periodicals/convergence/john-napier-his-life-his-logs-and-his-bones}
\item \link[Logarithms: The Early History of a Familiar Function]{https://www.maa.org/press/periodicals/convergence/logarithms-the-early-history-of-a-familiar-function-john-napier-introduces-logarithms}
\item \link[Johannes Kepler]{https://en.wikipedia.org/wiki/Johannes_Kepler}
\end{itemize}


\section{Readings}
First reading: Dunham, Chapter 7, pages 155 - 174

Second reading: \link[How Can We Multiply Quickly?]{https://www.youtube.com/watch?v=LcJPGsWErdQ}



\section{Questions}

\begin{question}
The video considers the multiplication problem $\answer[given]{17} \times 37$.
\end{question}

\begin{question}
Which of the following is not mentioned in the video as a way to speed up multiplication?
\begin{multipleChoice}
\choice[correct]{Using Roman Numerals.}
\choice {Using quarter squares.}
\choice {Using logarithms.}
\choice {Using a slide rule.}
\end{multipleChoice}
\end{question}


%\begin{question}
%What are the most important points from this reading?
%\begin{freeResponse}
%\end{freeResponse}
%
%\end{question}




\end{document}
