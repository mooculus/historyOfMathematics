\documentclass[nooutcomes]{ximera}

\title{Fermat's Last Theorem}

\begin{document}
\begin{abstract}
    
\end{abstract}
\maketitle

As we consider the Great Theorem of this chapter in our first reading, we'll also talk a bit more about one of the mathematicians we have mentioned a few times this semester: Pierre de Fermat.  A little of Fermat's life story was mentioned in the first reading from Dunham in this chapter, as well as his famous ``Last Theorem''.  Our second reading does a good job of explaining the basic ideas of the theorem and the history of its proof.  Notice that Dunham's text was published before Wiles' proof!  So, Dunham is no longer correct that this is an unsolved problem in mathematics.

The article is still a little older than I'd like, however!  Here is a bit more information to bring our story into the present.
\begin{itemize}
	\item Since the article above was published pretty close to the publication of Wiles' proof, it ends with some uncertainty about the proof.  That uncertainty has been removed - the proof has been reviewed and accepted.
	\item Wiles has since won a number of awards for his proof.  Most recently, in 2016 he received the \link[Abel Prize]{http://www.abelprize.no/} in Mathematics.  
	\item Wiles' proof was essentially to verify a relevant case of the Taniyama-Shimura-Weil conjecture.  The full conjecture was proven in 2001, and is usually referred to as the ``modularity theorem".  The \link[MathWorld article]{http://mathworld.wolfram.com/Taniyama-ShimuraConjecture.html} says a little more.
	\item Wiles originally gave the proof in about three lectures; now there are instances of entire courses being devoted to the matter.  If you're interested in seeing more details for the proof, you can read \link[a very short summary]{https://www.math.wisc.edu/~boston/boston1.pdf}, \link[a relatively short summary]{http://math.stanford.edu/~lekheng/flt/gouvea.pdf}, or \link[an entire textbook]{https://www.math.wisc.edu/~boston/869.pdf}.  You will likely need to have had a significant algebra course to understand much of the proof.
	\item The \link[Wikipedia page]{https://en.wikipedia.org/wiki/Fermat's_Last_Theorem} for Fermat's Last Theorem is heavily referenced and also not a bad place to get general information about the theorem.  (I know, I said you have to be careful with Wikipedia!  Be sure to double-check the information in a non-open-source location.)
\end{itemize}

\section{Readings}
First reading: Dunham, Chapter 7, pages 174 - 183

Second reading: \link[Fermat's Last Theorem]{http://www-history.mcs.st-andrews.ac.uk/HistTopics/Fermat's_last_theorem.html}



\section{Questions}

\begin{question}
Into how many cases did Sophie Germain divide the problem? $\answer[given]{2d}$
\end{question}

\begin{question}
Where was Wiles working when he solved Fermat's Last Theorem?
\begin{multipleChoice}
\choice{Stanford.}
\choice [correct]{Princeton.}
\choice {Cambridge.}
\choice {Oxford.}
\end{multipleChoice}
\end{question}


\begin{question}
What are the most important points from this reading?
\begin{freeResponse}
\end{freeResponse}

\end{question}




\end{document}
