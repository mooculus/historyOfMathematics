\documentclass[nooutcomes]{ximera}
\graphicspath{{./}{thePythagoreanTheorem/}{deMoivreSavesTheDay/}{complexNumbersFromDifferentAngles/}{trianglesOnACone/}{cityGeometry/}{EuclidAndGeometry/}}

\usepackage{gensymb}
\usepackage[margin=1in]{geometry}

%\usepackage{hyperref}


\usepackage{tikz}
\usepackage{tkz-euclide}
\usetkzobj{all}
\tikzstyle geometryDiagrams=[ultra thick,color=blue!50!black]
\newcommand{\tri}{\triangle}
\renewcommand{\l}{\ell}
\renewcommand{\P}{\mathcal{P}}
\newcommand{\R}{\mathbb{R}}
\newcommand{\Q}{\mathbb{Q}}

\newcommand{\Z}{\mathbb Z}
\newcommand{\N}{\mathbb N}
\newcommand{\ph}{\varphi}

\renewcommand{\vec}{\mathbf}
\renewcommand{\d}{\,d}


%\counterwithin*{question}{section} <- This didn't work



%% Egyptian symbols

\usepackage{multido}
\newcommand{\egmil}[1]{\multido{\i=1+1}{#1}{\includegraphics[scale=.1]{egyptian/egypt_person.pdf}\hspace{0.5mm}}}
\newcommand{\eghuntho}[1]{\multido{\i=1+1}{#1}{\includegraphics[scale=.1]{egyptian/egypt_fish.pdf}\hspace{0.5mm}}}
\newcommand{\egtentho}[1]{\multido{\i=1+1}{#1}{\includegraphics[scale=.1]{egyptian/egypt_finger.pdf}\hspace{0.5mm}}}
\newcommand{\egtho}[1]{\multido{\i=1+1}{#1}{\includegraphics[scale=.1]{egyptian/egypt_lotus.pdf}\hspace{0.5mm}}}
\newcommand{\eghun}[1]{\multido{\i=1+1}{#1}{\includegraphics[scale=.1]{egyptian/egypt_scroll.pdf}\hspace{0.5mm}}}
\newcommand{\egten}[1]{\multido{\i=1+1}{#1}{\includegraphics[scale=.1]{egyptian/egypt_heel.pdf}\hspace{0.5mm}}}
\newcommand{\egone}[1]{\multido{\i=1+1}{#1}{\includegraphics[scale=.1]{egyptian/egypt_stroke.pdf}\hspace{0.5mm}}}
\newcommand{\egyptify}[7]{
 \multido{\i=1+1}{#1}{\includegraphics[scale=.1]{egyptian/egypt_person.pdf}\hspace{0.5mm}}
 \multido{\i=1+1}{#2}{\includegraphics[scale=.1]{egyptian/egypt_fish.pdf}\hspace{0.5mm}}
 \multido{\i=1+1}{#3}{\includegraphics[scale=.1]{egyptian/egypt_finger.pdf}\hspace{0.5mm}}
 \multido{\i=1+1}{#4}{\includegraphics[scale=.1]{egyptian/egypt_lotus.pdf}\hspace{0.5mm}}
 \multido{\i=1+1}{#5}{\includegraphics[scale=.1]{egyptian/egypt_scroll.pdf}\hspace{0.5mm}}
 \multido{\i=1+1}{#6}{\includegraphics[scale=.1]{egyptian/egypt_heel.pdf}\hspace{0.5mm}}
 \multido{\i=1+1}{#7}{\includegraphics[scale=.1]{egyptian/egypt_stroke.pdf}\hspace{0.5mm}}
 \hspace{.5mm}
}




\title{Gauss}

\begin{document}
\begin{abstract}
    
\end{abstract}
\maketitle



One of the downsides of using a textbook like Dunham's is that we only have time to talk about a small number of mathematicians.  At this point in history, we begin to see an increasing number of names you would recognize from your studies, both in mathematics as well as other subjects.  To put Gauss in the epilogue of a chapter about Euler is, in my opinion, a disservice to this great mathematician.  Some people even consider him to be the greatest mathematician since antiquity - greater even than Euler.  You should, of course, form your own opinions after doing these readings.

We begin with Dunham's biography of Gauss, and then our second reading gives more detail about the content of Gauss' {\em Disquisitiones Arithmeticae}.   Finally, an optional third video (from the Numberphile channel) gives more about the connection between constructible numbers and Fermat numbers which are prime. There is another video on this channel demonstrating how the heptadecagon is constructed if you are interested in that!


\section{Readings}
First reading: Dunham, Chapter 10, pages 235 - 244

Second reading: \link[Gauss's Disquisitiones Arithmeticae]{https://link-springer-com.proxy.lib.ohio-state.edu/article/10.1007/s00407-012-0105-x}
\begin{itemize}
 \item This is from a larger work about Gauss and Sophie Germain. Please read at least Section 6, pages 602 - 606.
\end{itemize}

Third reading (video): \link[Heptadecagon and Fermat Primes]{https://www.youtube.com/watch?v=oYlB5lUGlbw}



\section{Questions}

\begin{question}
How many articles (or paragraphs) does {\em  Disquisitiones Arithmeticae} have? $\answer[given]{366}$
\end{question}

\begin{question}
In the final section of {\em  Disquisitiones Arithmeticae} (famous for the 17-gon), what is Gauss primarily discussing?
\begin{multipleChoice}
\choice{Use of a compass.}
\choice {Complex numbers.}
\choice [correct]{Polynomials and their roots.}
\choice {Modular arithmetic.}
\end{multipleChoice}
\end{question}

%
%\begin{question}
%What are the most important points from this reading?
%\begin{freeResponse}
%\end{freeResponse}
%
%\end{question}




\end{document}
