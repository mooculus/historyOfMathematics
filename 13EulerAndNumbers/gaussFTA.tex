\documentclass[nooutcomes]{ximera}


\title{The Fundamental Theorem of Algebra}

\begin{document}
\begin{abstract}
    We investigate Gauss' first proof of the Fundamental Theorem of Algebra.
\end{abstract}
\maketitle

\section*{Gauss' First Proof}

In 1799, as his doctoral dissertation, Gauss gave this proof of the Fundamental Theorem of Algebra (FTA).

\begin{question}
    Remind me: what does the FTA say?
\end{question}

To work through this proof, we will need DeMoivre's Theorem, which says that
\[
(\cos(\theta) + i \sin(\theta))^n = \cos(n\theta) + i \sin(n\theta).
\]
\begin{question}
    Explain why DeMoivre's Theorem tells us how to find the $n$-th roots of any complex number.
\end{question}

First, Gauss writes his general polynomial (with real coefficients) as
\[
X = x^m + A x^{m-1} + B x^{m-2} + \dots + L x + M.
\]
He is trying to show that $X$ has either a real factor $x-r$ or a complex factor $x^2 - 2xr\cos(\phi) + r^2$.

\begin{question}
    Why is the above statement the same as saying $X$ has $x=r(\cos(\phi) + i\sin(\phi))$ as a root?
\end{question}

Now, Gauss plugs $x=r(\cos(\phi) + i\sin(\phi))$ in to $X$ and separates his result into its real part and its imaginary part, getting
\[
U = r^m\cos(m\phi) + A r^{m-1} \cos((m-1)\phi) + \dots + L r \cos (\phi) + M
\]
and
\[
T = r^m\sin(m\phi) + A r^{m-1} \sin((m-1)\phi) + \dots + L r \sin (\phi).
\]
\begin{question}
    How was DeMoivre's Theorem involved, here?  Which of $U$ and $T$ is the real part?  The imaginary part?
\end{question}

Gauss then proves directly that if $T=U=0$, then either $X$ is divisible by $x-r$ or $X$ is divisible by $x^2-2xr\cos(\phi) + r^2$.  We won't prove that here, but keep in mind that means we are trying to prove that the graphs of $T=0$ and $U=0$ have an intersection.

Plot $T=0$, $U=0$ and a very large circle of radius $R$ on the polar axes (i.e. $r$ and $\phi$).  As $R\to \infty$, the curves $T=0$ and $U=0$ look more and more like $\text{Re}(x^m)$ and $\text{Im}(x^m)$.

\begin{question}
    Why is this last statement true?  What does it have to do with DeMoivre's Theorem?
\end{question}

The upshot of the last point is: if we draw a circle with large enough radius $R$, and look at the pieces (more technically ``branches'') of $T=0$ and $U=0$ behave inside our circle, we find that the intersections of the branches with the circle itself must alternate.

\begin{question}
    Why does the horizontal axis ($\phi = 0$ and $\phi = \pi$) have to be a branch of $T$?
\end{question}

On the circle below, sketch the horizontal axis and label it $T=0$.  Since each branch must both enter and leave the circle, label an even number of $U$-intersections, alternating with an even number of $T$-intersections.  Be sure to label which is which!

\begin{center}
    \begin{tikzpicture}
        \draw[thick] (0,0) circle (1.2in);
    \end{tikzpicture}
\end{center}

\begin{question}
    Why are we now forced to have an intersection of one of the branches of $T$ with one of the branches of $U$? (It may help to use two different colors!)
\end{question}

That's it!

\section*{d'Alembert's Proof}

Here is a theorem now known as {\em d'Alembert's Lemma}:
\begin{lemma}
If $p(z)$ is a polynomial function with $p(z_0) \neq 0$ for some $z_0$, then any neighborhood of $z_0$ contains a point $z_1$ such that $\vert p(z_1) \vert < \vert p(z_0) \vert$.
\end{lemma}

\begin{question}
    What is this lemma saying?
\end{question}

We'll also need the Extreme Value Theorem from calculus (sometimes called the Weierstrass Extreme Value Theorem).
\begin{question}
    Remind me: what does the Extreme Value Theorem say about a continuous function on a closed interval?
\end{question}

On to the proof of the FTA.  Step 1: We can find some radius $R$ so that for all $\vert z \vert \geq R$, $\vert p(z) \vert$ is an increasing function.

\begin{question}
    What is Step 1 saying?  Why is it true? (Hint: what does the graph of a degree-$m$ polynomial look like?  What about the absolute value of that polynomial?
\end{question}

Step 2: We know that $\vert p(z) \vert$ has to have an absolute minimum inside the circle.
\begin{question}
    What is Step 2 saying?  Why is it true?
\end{question}

Step 3: Suppose the minimum is strictly greater than zero.
\begin{question}
    If the minimum is strictly inside the circle, why does d'Alembert's lemma give us a contradiction?
\end{question}

\begin{question}
    If the minimum is on the circle's boundary, why do we have a contradiction?
\end{question}

\begin{question}
    Why have we now proven the FTA?
\end{question}

\end{document}