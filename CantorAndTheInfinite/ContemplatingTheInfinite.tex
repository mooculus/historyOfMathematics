\documentclass[nooutcomes]{ximera}

\title{Contemplating the Infinite}

\begin{document}
\begin{abstract}
    
\end{abstract}
\maketitle

If you've never thought much about the idea of ``infinity'' before, then the Great Theorems of Chapters 11 and 12 might be a little surprising.  If you've seen them before, it's always good to stop and think a bit about why these theorems are surprising.  The historical context in the readings in Dunham should help.

For fun, you might try to explain to a (non-mathematician) friend that there are infinite sets which are the same ``size'', and infinite sets which are different ``sizes''.  You'll first have to explain what you mean by ``size'', of course!  The second reading might help your explanation, since it can be used effectively in a high school classroom for talking about this topic.




\section{Readings}
First reading: Dunham, Chapters 11 and 12

Second reading: \link[The Infinite Hotel]{http://www.jstor.org.proxy.lib.ohio-state.edu/stable/20876420}.



\section{Questions}

\begin{question}
After the hotel appears to be full, a family arrives and asks for a room.  After moving the patrons around, which room is the new family assigned?
$\answer[given]{1}$
\end{question}

\begin{question}
In the story of ``Hotel Infinity'', what does George build, other than the hotel?
\begin{multipleChoice}
\choice[correct]{A fleet of buses.}
\choice {A courtroom.}
\choice {Billboards.}
\choice {An infinite parking garage.}
\end{multipleChoice}
\end{question}


\begin{question}
What are the most important points from this reading?
\begin{freeResponse}
\end{freeResponse}

\end{question}




\end{document}
