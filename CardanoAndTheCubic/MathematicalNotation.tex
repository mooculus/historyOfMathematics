\documentclass[nooutcomes]{ximera}

\title{Mathematical Notation}

\begin{document}
\begin{abstract}
    
\end{abstract}
\maketitle

As we consider Cardano's proof of our Great Theorem, one important concept to keep in mind is that much of our modern algebraic notation has not yet been developed.  You can see an example in the text on page 146.  An important following question is then: when was our modern algebraic notation developed, and by who?  The second reading sheds some light on this question.  An optional third reading gives a concise list of sources for the earliest uses of some common symbols.


\section{Readings}
First reading: Dunham, Chapter 6, pages 142 - 154

Second reading: \link[A Brief History of Algebraic Notation]{http://onlinelibrary.wiley.com.proxy.lib.ohio-state.edu/doi/10.1111/j.1949-8594.2000.tb17262.x/epdf}

Third reading: \link[Earliest Uses of Various Mathematical Symbols]{http://jeff560.tripod.com/mathsym.html}

\section{Questions}

\begin{question}
In the Greek system, what was the value of the symbol $\pi$? $\answer[given]{80}$
\end{question}

\begin{question}
What kind of algebra was Cardano using in his proof?
\begin{multipleChoice}
\choice{Rhetorical algebra.}
\choice {Syncopated algebra.}
\choice {Modern algebra.}
\choice [correct]{An early form of symbolic algebra.}
\end{multipleChoice}
\end{question}


\begin{question}
What are the most important points from this reading?
\begin{freeResponse}
\end{freeResponse}

\end{question}




\end{document}
