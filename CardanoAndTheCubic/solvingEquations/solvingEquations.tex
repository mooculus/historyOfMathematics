\documentclass{ximera}

\graphicspath{{./}{thePythagoreanTheorem/}{deMoivreSavesTheDay/}{complexNumbersFromDifferentAngles/}{trianglesOnACone/}{cityGeometry/}{EuclidAndGeometry/}}

\usepackage{gensymb}
\usepackage[margin=1in]{geometry}

%\usepackage{hyperref}


\usepackage{tikz}
\usepackage{tkz-euclide}
\usetkzobj{all}
\tikzstyle geometryDiagrams=[ultra thick,color=blue!50!black]
\newcommand{\tri}{\triangle}
\renewcommand{\l}{\ell}
\renewcommand{\P}{\mathcal{P}}
\newcommand{\R}{\mathbb{R}}
\newcommand{\Q}{\mathbb{Q}}

\newcommand{\Z}{\mathbb Z}
\newcommand{\N}{\mathbb N}
\newcommand{\ph}{\varphi}

\renewcommand{\vec}{\mathbf}
\renewcommand{\d}{\,d}



%\makeatletter
%\let\c@problem\relax
%\makeatother
%
%\let\problem\relax
%\let\endproblem\relax
%
%\newtheoremstyle{SlantTheorem}{\topsep}{\fill}%%% space between body and thm
% {\slshape}                      %%% Thm body font
% {}                              %%% Indent amount (empty = no indent)
% {\bfseries\sffamily}            %%% Thm head font
% {}                              %%% Punctuation after thm head
% {3ex}                           %%% Space after thm head
% {\thmname{#1}\thmnumber{ #2}\thmnote{ \bfseries(#3)}} %%% Thm head spec
%\theoremstyle{SlantTheorem}
%\newtheorem{problem}{Problem}[]
%
%
%\makeatletter
%\let\c@question\relax
%\makeatother
%
%\let\question\relax
%\let\endquestion\relax
%
%\newtheoremstyle{SlantTheorem}{\topsep}{\fill}%%% space between body and thm
% {\slshape}                      %%% Thm body font
% {}                              %%% Indent amount (empty = no indent)
% {\bfseries\sffamily}            %%% Thm head font
% {}                              %%% Punctuation after thm head
% {3ex}                           %%% Space after thm head
% {\thmname{#1}\thmnumber{ #2}\thmnote{ \bfseries(#3)}} %%% Thm head spec
%\theoremstyle{SlantTheorem}
%\newtheorem{question}{Question}[]






%\counterwithin*{question}{section} <- This didn't work



%% Egyptian symbols

\usepackage{multido}
\newcommand{\egmil}[1]{\multido{\i=1+1}{#1}{\includegraphics[scale=.1]{egyptian/egypt_person.pdf}\hspace{0.5mm}}}
\newcommand{\eghuntho}[1]{\multido{\i=1+1}{#1}{\includegraphics[scale=.1]{egyptian/egypt_fish.pdf}\hspace{0.5mm}}}
\newcommand{\egtentho}[1]{\multido{\i=1+1}{#1}{\includegraphics[scale=.1]{egyptian/egypt_finger.pdf}\hspace{0.5mm}}}
\newcommand{\egtho}[1]{\multido{\i=1+1}{#1}{\includegraphics[scale=.1]{egyptian/egypt_lotus.pdf}\hspace{0.5mm}}}
\newcommand{\eghun}[1]{\multido{\i=1+1}{#1}{\includegraphics[scale=.1]{egyptian/egypt_scroll.pdf}\hspace{0.5mm}}}
\newcommand{\egten}[1]{\multido{\i=1+1}{#1}{\includegraphics[scale=.1]{egyptian/egypt_heel.pdf}\hspace{0.5mm}}}
\newcommand{\egone}[1]{\multido{\i=1+1}{#1}{\includegraphics[scale=.1]{egyptian/egypt_stroke.pdf}\hspace{0.5mm}}}
\newcommand{\egyptify}[7]{
 \multido{\i=1+1}{#1}{\includegraphics[scale=.1]{egyptian/egypt_person.pdf}\hspace{0.5mm}}
 \multido{\i=1+1}{#2}{\includegraphics[scale=.1]{egyptian/egypt_fish.pdf}\hspace{0.5mm}}
 \multido{\i=1+1}{#3}{\includegraphics[scale=.1]{egyptian/egypt_finger.pdf}\hspace{0.5mm}}
 \multido{\i=1+1}{#4}{\includegraphics[scale=.1]{egyptian/egypt_lotus.pdf}\hspace{0.5mm}}
 \multido{\i=1+1}{#5}{\includegraphics[scale=.1]{egyptian/egypt_scroll.pdf}\hspace{0.5mm}}
 \multido{\i=1+1}{#6}{\includegraphics[scale=.1]{egyptian/egypt_heel.pdf}\hspace{0.5mm}}
 \multido{\i=1+1}{#7}{\includegraphics[scale=.1]{egyptian/egypt_stroke.pdf}\hspace{0.5mm}}
 \hspace{.5mm}
}




\title{Solving equations}
\begin{document}
\begin{abstract}
In this activity we will solve second and third degree equations.
\end{abstract}
\maketitle


Finding roots of quadratic polynomials is somewhat complex. We want to
find $x$ such that
\[
ax^2 + bx + c = 0.
\]
I know you already know how to do this. However, pretend for a moment
that you don't. This would be a really hard problem. We have evidence
that it took humans around 1000 years to solve this problem in
generality, the first general solution appearing in Babylon and China
around 2500 years ago. With this in mind, I think this topic warrants
some attention. If you want to solve $ax^2 + bx + c = 0$, a good place
to start would be with an easier problem. Let's make $a=1$ and try to
solve
\[
x^2 + b x = c
\]
Geometrically, you could visualize this as an $x \times x$ square
along with a $b\times x$ rectangle. Make a blob for $c$ on the other side. 

\begin{question} What would a picture of this look like?
\end{question}


\begin{question} What is the total area of the shapes in your picture?
\end{question}


Take your $b\times x$ rectangle and divide it into two
$(b/2)\times x$ rectangles.

\begin{question} What would a picture of this look like?
\end{question}


\begin{question} What is the total area of the shapes in your picture?
\end{question}


Now take both of your $(b/2)\times x$ rectangles and snuggie them
next to your $x\times x$ square on adjacent sides. You should now have
what looks like an $(x + \frac{b}{2}) \times (x +
\frac{b}{2})$ square with a corner cut out of it.


\begin{question} What would a picture of this look like?
\end{question}


\begin{question} What is the total area of the shapes in your picture?
\end{question}


Finally, your big $(x + \frac{b}{2}) \times (x + \frac{b}{2})$ has a
piece missing, a $(b/2) \times (b/2)$ square, right? So if you add
that piece in on both sides, the area of both sides of your picture
had better be $c + (b/2)^2$. From your picture you will find that:
\[
\left(x + \frac{b}{2}\right)^2 = c + \left(\frac{b}{2}\right)^2
\]

\begin{question} 
Can you find $x$ at this point?
\end{question}


\begin{question}
Explain how to solve $ax^2 + bx + c = 0$.
\end{question}



\subsection*{Cubic Equations}

While the quadratic formula was discovered around 2500 years ago,
cubic equations proved to be a tougher nut to crack. A general
solution to a cubic equation was not found until the 1500's. At the
time mathematicians were a secretive and competitive bunch. Someone
would solve a particular cubic equation, then challenge another
mathematician to a sort of ``mathematical duel.'' Each mathematician
would give the other a list of problems to solve by a given date. The
one who solved the most problems was the winner and glory
everlasting\footnote{This might be a slight exaggeration.}  was
theirs. One of the greatest duelists was Niccol\`{o} Fontana Tartaglia
(pronounced \textit{Tar-tah-lee-ya}). Why was he so great? He
developed a general method for solving cubic equations! However,
neither was he alone in this discovery nor was he the first. As
sometimes happens, the method was discovered some years earlier by
another mathematician, Scipione del Ferro. However, due to the secrecy
and competitiveness, very few people knew of Ferro's method. Since these
discoveries were independent, we'll call the method the
\textit{Ferro-Tartaglia method}.

We'll show you the Ferro-Tartaglia method\index{Ferro-Tartaglia
  method} for finding at least one root of a cubic of the form:
\[
x^3+ px + q
\]
We'll illustrate with a specific example---you'll have to generalize
yourself! Take
\[
x^3 +3x - 2  = 0
\]
All I can tell you are these three steps:
\begin{enumerate}
\item Replace $x$ with $u+v$. 
\item Set $uv$ so that all of the terms are eliminated except for $u^3$,
$v^3$, and constant terms.  
\item Clear denominators and use the quadratic formula.
\end{enumerate}


\begin{question} How many solutions are we supposed to have in total?
\end{question}

\begin{question}
Use the Ferro-Tartaglia method to solve $x^3 + 9x -26 =0$.
\end{question}


\begin{question}
How many solutions should our equation above have? Where/what are they?
Hint: Make use of an old forgotten foe\dots
\end{question}




\begin{question} How do we do this procedure for other equations of the form
\[
x^3 + px + q = 0?
\]
\end{question}

\end{document}
